\chapter*{Parte Objetiva}
\label{sec:parte_objetiva}
%% --------------------------------------------------------------------- %%
\chapter{Introdução}
\label{sec:intr}

É comum, principalmente no desenvolvimento de jogos eletrônicos, a necessidade
de um método prático de se inserir conteúdo lógico dinamicamente em uma
aplicação. Linguagens de script (como Lua ou Python) são ferramentas que, dentre
outras coisas, fornecem esse tipo de funcionalidade. Podem ter seu código fonte
tanto compilado e processado dinamicamente quanto incorporado na aplicação
desejada para que esta, por sua vez, adquira acesso a essas conveniências. No
entanto, tal incorporação pode trazer vários desafios, especialmente o de ter
que aprender a usar uma API (Application Programming Interface) diferente para
cada linguagem de script usada.

Esta monografia apresentará um sistema de software que permite ao usuário
integrar facilmente diversas linguagens de script a um programa escrito em
\verb$C++$ usando uma API unificada para incorporação e um exportador
automatizado de interfaces \verb$C++$ para essas linguagens de script. Ela
explicará todos os principais mecanismos desse sistema, assim como a teoria e a
tecnologia por trás deles.

\section{Motivações e Objetivos}
\label{sec:intr:motivacoes_objetivos}

\begin{enumerate}
  \item Desenvolvimento de uma biblioteca C++ que abstrai o uso da API de uma(s)
        linguagem(s) de script(s) de forma clara, prática de usar, e facilmente
        estendível para aceitar outras linguagens de script, sendo que
        inicialmente terá suporte a Lua e Python. O usuário da biblioteca não
        necessitará de conhecimento específico da API das linguagens para
        incorporá-las em seu programa, assim como não precisará se importar com
        qual linguagem de fato está sendo usada, pois o sistema será capaz de
        reconhecer as linguagens às quais ele fornece suporte.
  \item Desenvolvimento de uma ferramenta que interpreta parte do código C++ do
        usuário a fim de gerar wrappers (código fonte, nesse caso em C++, que
        serve de intermediário para um ambiente externo, como as linguagens de
        script) das classes e funções desse código para cada linguagem de script
        que o sistema estiver configurado para reconhecer. Assim, o usuário
        poderá usufruir das funcionalidades que desenvolveu em C++ na liguagem
        de script de sua preferência.
  \item Desenvolvimento de uma ferramenta que interpreta parte do código C++ do
        usuário a fim de gerar wrappers (código fonte, nesse caso em C++, que
        serve de intermediário para um ambiente externo, como as linguagens de
        script) das classes e funções desse código para cada linguagem de script
        que o sistema estiver configurado para reconhecer. Assim, o usuário
        poderá usufruir das funcionalidades que desenvolveu em C++ na liguagem
        de script de sua preferência.
  \item Desenvolvimento de qualquer outra ferramenta que seja necessária para o
        devido funcionamento conjunto dos dois resultados acima de maneira a
        simplificar o máximo possível o uso do sistema como um todo.
\end{enumerate}

\input cap-estrutura

\input cap-atividades

\input cap-resultados
