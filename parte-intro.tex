
\chapter{Introdução}
\label{cap:intr}

Programas de computador frequentemente lidam com diversos tipos de conteúdo:
textos, imagens, músicas, vídeos, etc. Eles são armazenados e manipulados na
forma de dados, segundo codificações específicas. Alguns tipos de conteúdo são
mais complexos do que outros, como o banco de dados da sua rede social favorita,
ou um registro de tomografias médicas. Algumas vezes dados são tão
complicados que se tornam programas por sí sós. É um caso bastante comum em
jogos eletrônicos, onde parte do conteúdo é a inteligência artifical dos
personagens ou o roteiro de eventos que ocorre para cada interação do jogador.
\textbf{Linguagens de \script{}} são ferramentas bastante populares para tratar
conteúdo desse gênero.

Para tanto, existem diversas ferramentas disponíveis no mercado e na comunidade
Web. Mas cada uma delas tem suas especifidades e restrições, dificilmente
trabalhando com a noção de \script{s} de forma mais geral, e ainda mais
raramente de maneira automatizada. Por isso, a triste realidade é que integrar
\script{s} em sua aplicação, por mais valor que isso agregue ao seu conteúdo,
exige muito esforço.

Portanto, para nosso trabalho de formatura, apresentamos uma solução em
\textit{software} que permite ao usuário integrar de forma automatizada as
linguagens de \script{} \lang{Lua} e \lang{Python} a um programa escrito em
\CXX{}. Apesar de a princípio nos restringirmos apenas a essas duas linguagens
de \script{}, elaboramos a arquitetura do nosso sistema de modo a facilitar
a contemplação de novas linguagens. A compatibilidade inicial com duas
linguagens diferentes de maneira simultânea visa demonstrar esse fato. Nesta
monografia, são explicados todos os principais mecanismos desse sistema, assim
como a teoria e a tecnologia por trás deles.

\section{Motivação}
\label{cap:intr:motivacao}

Como indicado por um exemplo altamente viesado acima, uma das nossas principais
motivações para esse trabalho é o desenvolvimento de jogos digitais. Mais
especificamente, foi nosso trabalho no USPGameDev\footnote{
  O USPGameDev é um grupo de pesquisa e desenvolvimento de jogos da
  Universidade de São Paulo. Ver \url{uspgamedev.org} (último acesso: 
  15/09/2013).
} que nos levou a ele. Afinal, em uma equipe de desenvolvedores de jogos não
há apenas programadores, mas também \textit{designers} e artistas, que pouco
ou nada sabem sobre programação. Nada melhor do que usar linguagens de
\script{}, conhecidas por sua expressividade e facilidade de uso, para
simplificar o trabalho deles e aumentar a produtividade geral do grupo.

O problema é que não conseguimos decidir qual linguagem de \script{} usar.
Por um lado tínhamos \lang{Python}, com suas vastas extensões e facilidades, e
por outro \lang{Lua}, com sua simplicidade e versatilidade. Depois de discutir
bastante e considerar as possibilidades, pensamos em uma idéia ``melhor de dois
mundos'': comportar ambas as linguagens, e ocultar os mecanismos internos de
cada uma de maneira que os outros desenvolvedores do grupo não precisassem se
preocupar com qual estivesse sendo usada de fato. Além disso, como programadores
frescos que somos, queríamos que o sistema cuidasse de todo o trabalho sujo
envolvido e tivesse uma interface limpa e elegante de usar. Como contaremos mais
adiante, isso tudo ocorreu a dois anos atrás quando nós dois, autores deste
trabalho, tornamo-nos os responsáveis por esse projeto dentro do USPGameDev.

Conseguimos obter resultados satisfatoriamente funcionais com alguns meses de
desenvolvimento. No entanto, no começo desse ano ainda havia vários aspectos que
queríamos melhorar, principalmente devido às limitações impostas por uma das
ferramentas que usávamos. Por isso, decidimos transformar esse sistema em nosso
trabalho de formatura, dando continuidade aos aprimoramentos e registrando
nossos aprendizados e experiências na monografia.

\section{Objetivos}
\label{cap:intr:objetivos}

  O objetivo desse nosso projeto, contando com o que já desenvolvemos antes de
  ele se tornar oficalmente nosso trabalho de formatura, consiste no
  desenvolvimento de um sistema de \textit{software} capaz de integrar
  aplicações desenvolvidas em \CXX{} com \script{s} escritos em \lang{Lua} e
  \lang{Python} mantendo as seguintes características:
  
  \begin{enumerate}

    \item A aplicação deverá ser capaz de acessar \script{s} de maneira simples
          o suficiente para que ela não precise se referir diretamente aos
          mecanismos internos das linguagens envolvidas ou até mesmo saber qual a
          linguagem com que os \script{s} foram escritos.
    \item O sistema cuidará de disponibilizar as funcionalidades relevantes da
          aplicação para os \script{s} que forem usados por ela.
    \item O sistema será um \textit{Software Livre}, e portanto deverá ser de
          fácil distribuição e inclusão nas aplicações que desejarem seus
          serviços.

  \end{enumerate}

