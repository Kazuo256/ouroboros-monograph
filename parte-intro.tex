
\chapter{Introdução}
\label{sec:intr}

É comum, principalmente no desenvolvimento de jogos eletrônicos, a necessidade
de um método prático de se inserir conteúdo lógico dinâmicamente em uma
aplicação (isto é, durante a execução dela). Linguagens de \script{} (como Lua ou
Python) são ferramentas popularmente usadas para atender esse tipo de
funcionalidade. Elas podem ter seu código fonte tanto compilado e processado
dinamicamente quanto incorporado na aplicação desejada. A aplicação, por sua
vez, torna-se capaz de acessar essas conveniências. No entanto, tal incorporação
pode trazer vários desafios, especialmente o de ter que aprender a usar uma API
(\textit{Application Programming Interface}) diferente para cada linguagem de
\script{} usada.

Esta monografia apresentará um sistema de software que permite ao usuário
integrar facilmente diversas linguagens de \script{} a um programa escrito em
\CXX{} usando uma API unificada para incorporação e um exportador
automatizado de interfaces em \CXX{} para essas linguagens de \script{}. São
explicados todos os principais mecanismos desse sistema, assim como a teoria e a
tecnologia por trás deles.

\section{Motivação}
\label{sec:intr:motivacao}

\section{Objetivos}
\label{sec:intr:objetivos}

Pode-se resumir a proposta do nosso projeto nas seguintes partes:

\begin{enumerate}
  \item Desenvolvimento de uma biblioteca \CXX{} que abstrai o uso da API de uma(s)
        linguagem(s) de \script{} de forma clara, prática de usar, e facilmente
        estendível para aceitar outras linguagens de \script{}, sendo que
        inicialmente terá suporte a Lua e Python. O usuário da biblioteca não
        necessitará de conhecimento específico da API das linguagens para
        incorporá-las em seu programa, assim como não precisará se importar com
        qual linguagem de fato está sendo usada, pois o sistema será capaz de
        reconhecer as linguagens às quais ele fornece suporte.
  \item Desenvolvimento de uma ferramenta que interpreta parte do código \CXX{} do
        usuário a fim de gerar wrappers (código fonte, nesse caso em \CXX{}, que
        serve de intermediário para um ambiente externo, como as linguagens de
        \script{}) das classes e funções desse código para cada linguagem de \script{}
        que o sistema estiver configurado para reconhecer. Assim, o usuário
        poderá usufruir das funcionalidades que desenvolveu em \CXX{} na liguagem
        de \script{} de sua preferência.
  \item Desenvolvimento de módulos de \lang{CMake} que garantam a devida
        compilação e funcionamento conjunto dos dois resultados acima de maneira
        a simplificar o máximo possível o uso do sistema como um todo. Também
        elaborar uma documentação básica sobre como usar o \emph{software}
        resultante do projeto.
\end{enumerate}
