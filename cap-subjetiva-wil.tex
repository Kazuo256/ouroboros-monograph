%% ------------------------------------------------------------------------- %%
\chapter{Wilson Kazuo Mizutani}
\label{cap:wil_subjetiva}

\caption{Desafios e Frustrações}
\label{cap:wil:desafios_frustracoes}

Isso será escrito na versão final da monografia.

\caption{Relação entre o trabalho de formatura e disciplinas do BCC}
\label{cap:wil:relacao_disciplinas_bcc}

\materia{MAC0122}{Princípios de Desenvolvimento de Algoritmos}{
  É a primeira matéria onde tive contado com \C{}, e na qual aprendi o
  essencial para desenvolver meus próprios algoritmos. Os exercícios-programa
  ajudaram muito a incentivar minha curiosidade como programador, assim como
  minha vontade por tentar entender mais a fundo algoritmos de diversos tipos
  e as maneiras de torná-los mais eficientes.
}

\materia{MAC0211}{Laboratório de Programação I}{
  Essencialmente pela experiência de fazer um projeto de médio porte, e pelas
  diversas ferramentas ensinadas ao longo da ementa: Makefile, LaTeX, flex,
  bison, principalemente.
}

\materia{MAC0323}{Estrutura de Dados}{
  Além de aprender a ideia por trás das estruturas de dados clássicas, um
  aspecto interessante da matéria foi que o professor ressaltou a importância
  de se fazer boas APIs. Ao longo de todo nosso trabalho, uma das minhas
  principais preocupações sempre foi garantir que o usuário do nosso sistema
  tivesse facilidade de entender e usar os serviçoes que forneceríamos.
}

\materia{MAC0316}{Conceitos Fundamentais de Linguagens de Programação}{

}

\materia{MAC0335}{Leitura Dramática}{

}

\materia{MAC0342}{Laboratório de Programação Extrema}{

}

\materia{MAC0414}{Linguagens Formais e Autômatos}{

}

\materia{MAC0424}{O Computador na Sociedade e Empresa}{

}

