\def\classname#1{\texttt{#1}}

\chapter{Atividades Realizadas}
\label{sec:atividades}

O Projeto Ouroboros não foi uma ideia que tivemos exclusivamente para nosso
trabalho de formatura. Como será relatado nas seções a seguir, ele começou como
parte de um projeto maior, mas que decidimos emancipar e incrementar para
apresentar neste trabalho. Por esse motivo, muito do que foi feito no
desenvolvimento deste sistema antes dessa decisão não levou em consideração a
possibilidade dele vir a se tornar independente do contexto a que pertencia.

Isso levou a uma série de questionamentos sobre como seria possível tornar nosso
sistema o mais completo e acessível possível por sí só, e quanto disso realmente
teríamos tempo de concretizar. É na solução desses problemas que focamos nossas
atenções durante o período do trabalho de formatura propriamente dito. O corpo
principal do projeto já estava garantido pelo nosso trabalho anterior a essa
etapa, quando ele ainda servia como componente de um sistema maior.

\section{Origem do projeto}
\label{sec:atividades:origem}

Eu e o Omar somos membros do USPGameDev. Lá, nós e o resto da equipe
desenvolvemos uma \emph{engine} para jogos bidimensionais, chamada UGDK. Ela é
feita em \CXX{}, uma linguagem de programação não exatamente simples. E aos
poucos isso se tornou um gargalo para o desenvolvimento de nossos jogos.
Decidimos que recisávamos de suporte a alguma linguagem de \script{}.

Mas não conseguimos chegar a um consenso de qual usar. Por um lado tínhamos
Python, com suas vastas extensões e facilidades, e por outro Lua, com sua
simplicidade minimalista e versátil. Então decidimos usar ambas.

Esse foi o começo do então chamado "sistema de \script{} da UGDK": um sistema
que premitiria ao desenvolvedor carregar módulos do seu jogo a partir de
\script{} sem que ele tivesse que se preocupar em determinar e tratar a
linguagem de origem. Descobrimos umas ferramenta chamada SWIG, que parecia ser
exatamente o que precisávamos. Com ele, consequimos completar uma versão inicial
do sistema de \script{}, atualmente ativo e funcional na UGDK.

Para o TCC, eu e o Omar resolvemos usar esse trabalho que fizemos. Separamos o
código dele da UGDK e fizemos nosso próprio repositório. Planejamos fazer
melhorias e reformulações do projeto, visando sempre tornar a vida mais fácil
para o usuário dessa nossa ferramenta.

%\classname{UmaClasse} 
%
%\begin{codebox}
%    \Procname{$\proc{XTR-mul}(x, y)$}
%    \zi $\id{temp} \gets x_1 * y_2 + x_2 * y_1$
%    \zi \Return $\proc{XTRElem}(x_2 * y_2 - \id{temp}, x_1 * y_1 -
%    \id{temp})$
%\end{codebox}

\section{Uma interface comum para Lua e Python.}
\label{sec:atividades:opa}

TODO

\section{Integração com SWIG}

TODO

\section{Problemas com SWIG}

TODO

\section{Nosso próprio gerador de \emph{wrappers}}

TODO

\section{Módulos de CMake}

TODO
