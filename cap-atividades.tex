\def\classname#1{\texttt{#1}}

\chapter{Atividades Realizadas}
\label{sec:atividades}

O sistema do Projeto Ouroboros é anterior ao nosso trabalho de formatura. Como
será relatado nas seções a seguir, ele começou como parte de um outro projeto.
Por esse motivo, a maior parte do desenvolvimento que fizemos nesse último ano
foi com o intuito de completar a emancipação dele como um \emph{software}
independente.

Fora isso, também houve dificuldades no uso de algumas ferramentas auxiliares
dos quais nosso sistema dependia. Elas impuseram certas limitações que deixaram
de ser aceitáveis quando decidimos que o projeto deveria ser capaz de satisfazer
usuários que não fossem nós mesmos.

\section{Origem do projeto}
\label{sec:atividades:origem}

%TODO "estes que vos escrevem"
%TODO Essa seção precisa ser beeeeeem mais detalhada.
Como membros do USPGameDev\footnote{Grupo de pesquisa e desenvolvimento de jogos
da USP. Ver \url{uspgamedev.org} (último acesso em 25/07/2013)},
fomos um dos vários contribuintes para a principal \emph{engine} de jogos
bidimensionais que o grupo vem desenvolvendo desde sua fundação, a
UGDK\footnote{\emph{USP Game Development Kit} ou \emph{USPGameDev Kit}. Ver
\url{uspgamedev.org/ugdk} (último acesso em 25/07/2013)}. Ela é programada em
\CXX{}, e no começo de 2011 estes que vos escrevem decidiram que seria
conveniente que ela tivesse suporte a alguma linguagem de \script{}. O principal
motivo para tanto era que um dos jogos sendo desenvolvidos pelo grupo, o
\emph{Horus Eye}\footnote{Ver \url{uspgamedev.org/horus-eye} (último acesso em
25/07/2013)}, usava essa \emph{engine} e estava ficando com seu código fonte
muito complexo, de tal maneira que era necessário muito esforço para acrescentar
conteúdo novo ao jogo. Colocando na UGDK a capacidade de trabalhar com
\script{s}, resolveríamos esse problema de uma maneira que os outros usuários
dela pudessem aproveitar.

Dessa maneira, surgiu a ideia inicial de programar um conjunto de
funcionalidades capaz de facilitar a comunicação entre uma linguagem compilada
(\CXX{}) e uma linguagem de \script{}. No entanto, não conseguimos chegar
imediatamente a um consenso sobre qual linguagem de \script{} usar
especificamente. As mais comunmente conhecidas na área de desenvolvimento de
jogos eletrônicos são Lua e Python. Então de um lado tínhamos as vastas
extensões e facilidades de Python, e do outro a simplicidade minimalista e
versátil de Lua. O melhor jeito de resolver o impasse foi usar ambas.

Isso levou ao então chamado "sistema de \script{s} da UGDK": um sistema que
premite que o desenvolvedor-usuário carregue módulos do seu jogo a partir de
\script{s} sem que ele tenha que se preocupar em determinar e tratar a linguagem
de origem deles. Com a ajuda de uma ferramenta que descobrimos nessa época
chamada SWIG, consequimos completar uma versão inicial do sistema de \script{s},
que existe até hoje no código da UGDK\footnote{Há planos para a UGDK usar
diretamente o sistema do Projeto Ouroboros quando ele estiver pronto,
abrindo mão do atual sistema de \script{s} próprio dela}.

%\classname{UmaClasse} 
%
%\begin{codebox}
%    \Procname{$\proc{XTR-mul}(x, y)$}
%    \zi $\id{temp} \gets x_1 * y_2 + x_2 * y_1$
%    \zi \Return $\proc{XTRElem}(x_2 * y_2 - \id{temp}, x_1 * y_1 -
%    \id{temp})$
%\end{codebox}

\section{Uma interface comum para Lua e Python.}
\label{sec:atividades:opa}

TODO

\section{Integração com SWIG}

TODO

\section{Problemas com SWIG}

TODO

\section{Nosso próprio gerador de \emph{wrappers}}

TODO

\section{Módulos de CMake}

TODO
