  % Arquivo LaTeX de exemplo de dissertação/tese a ser apresentados à CPG do IME-USP
% 
% Versão 5: Sex Mar  9 18:05:40 BRT 2012
%
% Criação: Jesús P. Mena-Chalco
% Revisão: Fabio Kon e Paulo Feofiloff
%  
% Obs: Leia previamente o texto do arquivo README.txt

% Alterado para monografia do trabalho de formatura (graduação).

\documentclass[11pt,twoside,a4paper]{book}

% ---------------------------------------------------------------------------- %
% Pacotes 
\usepackage[T1]{fontenc}
\usepackage[brazil]{babel}
\usepackage[utf8]{inputenc}
\usepackage[pdftex]{graphicx}           % usamos arquivos pdf/png como figuras
\usepackage{setspace}                   % espaçamento flexível
\usepackage{indentfirst}                % indentação do primeiro parágrafo
\usepackage{makeidx}                    % índice remissivo
\usepackage[nottoc]{tocbibind}          % acrescentamos a bibliografia/indice/conteudo no Table of Contents
\usepackage{courier}                    % usa o Adobe Courier no lugar de Computer Modern Typewriter
\usepackage{type1cm}                    % fontes realmente escaláveis
\usepackage{upquote}                    % arruma problema com aspas simples no listings
\usepackage{listings}                   % para formatar código-fonte (ex. em Java)
\usepackage{titletoc}
\usepackage{amsfonts}
\usepackage{amsmath}
\usepackage{clrscode}
\usepackage[sf,bf,small]{titlesec} % cabeçalhos dos títulos: menores e compactos
\usepackage[fixlanguage]{babelbib}
\usepackage[font=small,format=plain,labelfont=bf,up,textfont=it,up]{caption}
\usepackage{subcaption}
\usepackage{wrapfig}
\usepackage[usenames,svgnames,dvipsnames]{xcolor}
\usepackage[a4paper,top=2.54cm,bottom=2.0cm,left=2.0cm,right=2.54cm]{geometry} % margens
%\usepackage[pdftex,plainpages=false,pdfpagelabels,pagebackref,colorlinks=true,citecolor=black,linkcolor=black,urlcolor=black,filecolor=black,bookmarksopen=true]{hyperref} % links em preto
\usepackage[pdftex,plainpages=false,pdfpagelabels,pagebackref,colorlinks=true,citecolor=DarkGreen,linkcolor=NavyBlue,urlcolor=DarkRed,filecolor=green,bookmarksopen=true]{hyperref} % links coloridos
\usepackage[all]{hypcap}                % soluciona o problema com o hyperref e capitulos
\usepackage[square,sort,nonamebreak,comma]{natbib}  % citação bibliográfica alpha (alpha-ime.bst)
\usepackage{framed,color}
\fontsize{60}{62}\usefont{T1}{cmr}{m}{n}{\selectfont}

% ---------------------------------------------------------------------------- %
% Cabeçalhos similares ao TAOCP de Donald E. Knuth
\usepackage{fancyhdr}
\pagestyle{fancy}
\fancyhf{}
\renewcommand{\chaptermark}[1]{\markboth{\MakeUppercase{\texttt{#1}}}{}}
\renewcommand{\sectionmark}[1]{\markright{\MakeUppercase{\texttt{#1}}}{}}
\renewcommand{\headrulewidth}{0pt}

% ---------------------------------------------------------------------------- %
\graphicspath{{./figuras/}}             % caminho das figuras (recomendável)
\frenchspacing                          % arruma o espaço: id est (i.e.) e exempli gratia (e.g.) 
\urlstyle{same}                         % URL com o mesmo estilo do texto e não mono-spaced
\makeindex                              % para o índice remissivo
\raggedbottom                           % para não permitir espaços extra no texto
\fontsize{60}{62}\usefont{T1}{cmr}{m}{n}{\selectfont}
\cleardoublepage
\normalsize

% ---------------------------------------------------------------------------- %
% Opções de listing usados para o código fonte
% Ref: http://en.wikibooks.org/wiki/LaTeX/Packages/Listings
\lstdefinelanguage{lua}
  {morekeywords={and,break,do,else,elseif,end,false,for,function,if,in,local,
     nil,not,or,repeat,return,then,true,until,while},
   morekeywords={[2]arg,assert,collectgarbage,dofile,error,_G,getfenv,
     getmetatable,ipairs,load,loadfile,loadstring,next,pairs,pcall,print,
     rawequal,rawget,rawset,select,setfenv,setmetatable,tonumber,tostring,
     type,unpack,_VERSION,xpcall},
   morekeywords={[2]coroutine.create,coroutine.resume,coroutine.running,
     coroutine.status,coroutine.wrap,coroutine.yield},
   morekeywords={[2]module,require,package.cpath,package.load,package.loaded,
     package.loaders,package.loadlib,package.path,package.preload,
     package.seeall},
   morekeywords={[2]string.byte,string.char,string.dump,string.find,
     string.format,string.gmatch,string.gsub,string.len,string.lower,
     string.match,string.rep,string.reverse,string.sub,string.upper},
   morekeywords={[2]table.concat,table.insert,table.maxn,table.remove,
   table.sort},
   morekeywords={[2]math.abs,math.acos,math.asin,math.atan,math.atan2,
     math.ceil,math.cos,math.cosh,math.deg,math.exp,math.floor,math.fmod,
     math.frexp,math.huge,math.ldexp,math.log,math.log10,math.max,math.min,
     math.modf,math.pi,math.pow,math.rad,math.random,math.randomseed,math.sin,
     math.sinh,math.sqrt,math.tan,math.tanh},
   morekeywords={[2]io.close,io.flush,io.input,io.lines,io.open,io.output,
     io.popen,io.read,io.tmpfile,io.type,io.write,file:close,file:flush,
     file:lines,file:read,file:seek,file:setvbuf,file:write},
   morekeywords={[2]os.clock,os.date,os.difftime,os.execute,os.exit,os.getenv,
     os.remove,os.rename,os.setlocale,os.time,os.tmpname},
   alsodigit = {.},
   sensitive=true,
   morecomment=[l]{--},
   morecomment=[s]{--[[}{]]},
   morestring=[b]",
   morestring=[d]',
   morestring=[s]{[[}{]]},
  }

\lstset{ %
  language=C++,                   % choose the language of the code
  basicstyle=\footnotesize,       % the size of the fonts that are used for the code
  numbers=left,                   % where to put the line-numbers
  numberstyle=\footnotesize,      % the size of the fonts that are used for the line-numbers
  stepnumber=1,                   % the step between two line-numbers. If it's 1 each line will be numbered
  numbersep=5pt,                  % how far the line-numbers are from the code
  showspaces=false,               % show spaces adding particular underscores
  showstringspaces=false,         % underline spaces within strings
  showtabs=false,                 % show tabs within strings adding particular underscores
  frame=single,                 % adds a frame around the code
  framerule=0.6pt,
  tabsize=2,                      % sets default tabsize to 2 spaces
  captionpos=b,                   % sets the caption-position to bottom
  breaklines=true,                % sets automatic line breaking
  breakatwhitespace=false,        % sets if automatic breaks should only happen at whitespace
  escapeinside={\%*}{*)},         % if you want to add a comment within your code
  backgroundcolor=\color[rgb]{1.0,1.0,1.0}, % choose the background color.
  rulecolor=\color[rgb]{0.8,0.8,0.8},
  basicstyle=\ttfamily\scriptsize,
  keywordstyle=\color{blue}\bfseries,
  keywordstyle=[2]\color[rgb]{.4,0,.4}\bfseries,
  commentstyle=\color[rgb]{0,.6,0},
  stringstyle=\color{red},
  showstringspaces=false,
  upquote=true,
  extendedchars=true,
  xleftmargin=10pt,
  xrightmargin=10pt,
  framexleftmargin=15pt,
  framexrightmargin=10pt,
  morekeywords={[2]include,ifdef,define,ifndef,endif,nullptr}
}

% ---------------------------------------------------------------------------- %
% Corpo do texto
\begin{document}
\frontmatter 
% cabeçalho para as páginas das seções anteriores ao capítulo 1 (frontmatter)
\fancyhead[RO]{{\footnotesize\rightmark}\hspace{2em}\thepage}
\setcounter{tocdepth}{2}
\fancyhead[LE]{\thepage\hspace{2em}\footnotesize{\leftmark}}
\fancyhead[RE,LO]{}
\fancyhead[RO]{{\footnotesize\rightmark}\hspace{2em}\thepage}

\onehalfspacing  % espaçamento

% ---------------------------------------------------------------------------- %
% CAPA
% Nota: O título para as dissertações/teses do IME-USP devem caber em um 
% orifício de 10,7cm de largura x 6,0cm de altura que há na capa fornecida pela SPG.
\thispagestyle{empty}
\begin{center}
  \vspace*{2.3cm}
  \textsf{\textbf{
    {\LARGE Projeto Ouroboros }\\
    {
      \Large Sistema de integração automatizada entre \\
      \texttt{C++} e linguagens de \emph{script}
    }
  }}
    
  \vspace*{1.2cm}
  \textsf{\Large{Fernando Omar Aluani}} \\
  \textsf{\Large{Wilson Kazuo Mizutani}}
  
  \vskip 2cm
  \textsc{Trabalho de conclusão de curso} 
  
  \vskip 10cm
  \textsf{Orientador: Prof. Dr. Marcos Dimas Gubitoso}

  \vskip 3cm
  
  \normalsize{São Paulo, Dezembro de 2013}
\end{center}

\pagenumbering{roman}     % começamos a numerar 

% ------------------------------------------------------------------ %
% facilidades                                                        %
% ------------------------------------------------------------------ %

\renewcommand{\ttdefault}{txtt}
\def\cyclic#1{\langle #1 \rangle}
\def\script{\emph{script}}
\def\lang#1{\texttt{#1}}
\def\CXX{\lang{C++}}
\def\C{\lang{C}}
\def\VObj{\lang{VirtualObj}}
\def\SMgr{\lang{ScriptManager}}
\def\VMac{\lang{VirtualMachine}}
\def\VData{\lang{VirtualData}}
\def\str#1{cadeia#1 de caracteres}

% comandos novos %
\newcounter{defcnt}
\newcommand\definicao[1]{
    \stepcounter{defcnt}
    \begin{framed}
      \textbf{Definição \thechapter.\thedefcnt:}

      \vspace{.5em}
      \hspace{-20pt}
      #1
    \end{framed}
}
\newcommand\notacao{
    \textbf{Notação} \hspace{0.2cm}
}

% --------------------------------- %
% Agradecimentos
\chapter*{Agradecimentos}

Muitos de nós programadores sofremos de uma curiosidade mórbida em fazer
programas incrivelmente elaborados e inusitados idenpendentemente da utilidade
deles só para saber se uma ideia que tivemos realmente funciona. Muitas vezes é
esse impulso que leva ao desenvolvimento de sistemas importantes, outras ele
simplesmente leva a boas risadas. Mas uma coisa é certa: essas obsessões são boa
parte do que nos faz querer aprender mais e mais.

Gostaríamos de agradecer a todos que incentivaram em nós essa curiosidade
mórbida, tanto professores quanto amigos. Seja por ter-nos mostrado um truque
de código, ou ensinado alguma técnica secreta de programação, ou ainda
simplesmente apoiado uma de nossas ideias malucas. Muito do que fizemos nesse
trabalho vem da nossa vontade de experimentar com combinações divertidas do que
aprendemos ao longo da nossa vida como estudantes de Ciência da Computação.

% ---------------------------------------------------------------------------- %
% Resumo
\chapter*{Resumo}

Este trabalho é sobre uma biblioteca \CXX{} que estamos desenvolvendo desde
2011. O propósito dela é automatizar a integração entre aplicações programadas
em \CXX{} e \script{s} escritos em \lang{Lua} ou \lang{Python}. Normalmente,
o desenvolvedor da aplicação teria que escrever pelo menos algumas dúzias
de linhas de código para que ela pudesse simplesmente carregar um \script{}
para incorporar seus dados e rotinas. É justamente esse esforço adicional
qual queremos poupar ao usuário. Essa monografia fala sobre a pesquisa que
fizemos por ferramentas que nos ajudassem, sobre o processo de desenvolvimento
e sobre o produto final do nosso trabalho. Começamos por um capítulo introdutório,
seguida da parte objetiva, composta por três capítulos, e depois tratamos
a parte subjetiva, também dividida em três capítulos.

No primeiro capítulo (\ref{cap:intr} - Introdução), introduzimos a noção de
\script{ing} e a utilidade que ela tem em aplicações de computador. Usamos um
exemplo para ilustrar um caso de uso, e discutimos brevemente as possibilidades
atualmente existentes para integrar \script{s}. Depois, contamos sobre a
motivação que nos levou a elaborar esse projeto, assim como o que diferencia
ele das outras soluções presentes na comunidade Web. Terminamos listando os
objetivos principais do trabalho com relação à biblioteca de programação
que desenvolvemos.

Em seguida, partimos para uma discussão mais conceitual ao longo do próximo
capítulo (\ref{cap:conceitos} - Conceitos e tecnologias estudadas).
Procuramos entender o que caracteriza as linguagens de programação que estamos
tentando integrar: \CXX{}, \lang{Lua} e \lang{Python}. Para isso, exploramos
as noções de linguagens compiladas contra linguagens interpretadas, e como
o conceito de máquinas virtuais ajuda a entender melhor a relação entre elas.
Ao final, tratamos com mais profundidade sobre as ferramentas que \lang{Lua} e
\lang{Python} nos oferecem, para que fique mais fácil de entender o que
falamos nos demais capítulos do trabalho.

Depois, apresentamos a estrutura do nosso projeto (\ref{cap:estrutura} -
Estrutura do Projeto). Buscamos modelar uma solução para lidar com a
integração automatizada que desejamos que nosso sistema forneça. Esse
capítulo revela como dividimos nosso trabalho em duas grandes frentes,
que chamamos de \textbf{incorporação} e \textbf{exportação}. Tentamos
também deixar claro como o usuário poderá interagir com tudo isso.

Discutimos, no quarto capítulo (\ref{cap:atividades} - Atividades), sobre a
evolução do nosso sistema, desde sua origem a dois anos atrás até as decisões
que tomamos ao longo desse último ano tendo em vista o trabalho de formatura.
Contamos sobre como nossa participação no USPGameDev levou à ideia inicial
do projeto, assim como o que mudou quando o desvinculamos do grupo. Em seguida,
entramos em alguns detalhes da implementação do nosso sistema, segundo
as duas partes principais idealizadas no capítulo anterior. Também desabafamos sobre
o SWIG, uma ferramenta que nos ajudou inicialmente mas tornou-se uma
enorme dor de cabeça posteriormente. Terminamos falando sobre algumas utilidades
que fizemos em CMake para facilitar o uso da nossa biblioteca.

No último capítulo da parte objetiva (\ref{cap:resultados} - Resultados)
apresentamos o estado em que o projeto se encontra atualmente. Mostramos e
exemplificamos as funcionalidades, com alguns trechos de código para
deixar mais explícido o que ele é capaz de fazer. Aproveitamos para deixar
algumas instruções de uso, incluindo configuração e compilação das partes
relevantes do sistema.

A parte subjetiva começa com um capítulo para cada autor (\ref{cap:omar} - Fernando
Omar Aluani, e \ref{cap:wil} - Wilson Kazuo Mizutani), contando sobre nossas
dificuldades e frustrações no trabalho. Também apresentamos uma relação de
disciplinas que cursamos ao longo da nossa graduação que acreditamos ter nos
ajudado de uma maneira ou de outra. Encerramos a parte subjetiva da monografia
com um breve capítulo (\ref{cap:proximos_passos} - Próximos passos) listando
as nossas metas futuras para nosso projeto.

% ---------------------------------------------------------------------------- %
% Sumário
\tableofcontents    % imprime o sumário

% ---------------------------------------------------------------------------- %
% Capítulos do trabalho
\mainmatter

% cabeçalho para as páginas de todos os capítulos
\fancyhead[RE,LO]{\thesection}

\singlespacing              % espaçamento simples

\input parte-intro
\input parte-objetiva       % associado ao arquivo: 'cap-objetiva.tex'
\input parte-subjetiva      % associado ao arquivo: 'cap-subjetiva.tex'

% cabeçalho para os apêndices
\renewcommand{\chaptermark}[1]{\markboth{\MakeUppercase{\appendixname\ \thechapter}} {\MakeUppercase{#1}} }
\fancyhead[RE,LO]{}
\appendix

%\include{ape-conjuntos}      % associado ao arquivo: 'ape-conjuntos.tex'

% ---------------------------------------------------------------------------- %
% Bibliografia
\renewcommand\bibname{Referências}
\backmatter \singlespacing   % espaçamento simples
\bibliographystyle{alpha-ime}% citação bibliográfica alpha
\bibliography{bibliografia}  % associado ao arquivo: 'bibliografia.bib'

\end{document}
