% Arquivo LaTeX de exemplo de dissertação/tese a ser apresentados à CPG do IME-USP
% 
% Versão 5: Sex Mar  9 18:05:40 BRT 2012
%
% Criação: Jesús P. Mena-Chalco
% Revisão: Fabio Kon e Paulo Feofiloff
%  
% Obs: Leia previamente o texto do arquivo README.txt

% Alterado para monografia do trabalho de formatura (graduação).

\documentclass[11pt,twoside,a4paper]{book}

% ---------------------------------------------------------------------------- %
% Pacotes 
\usepackage[T1]{fontenc}
\usepackage[brazil]{babel}
\usepackage[utf8]{inputenc}
\usepackage[pdftex]{graphicx}           % usamos arquivos pdf/png como figuras
\usepackage{setspace}                   % espaçamento flexível
\usepackage{indentfirst}                % indentação do primeiro parágrafo
\usepackage{makeidx}                    % índice remissivo
\usepackage[nottoc]{tocbibind}          % acrescentamos a bibliografia/indice/conteudo no Table of Contents
\usepackage{courier}                    % usa o Adobe Courier no lugar de Computer Modern Typewriter
\usepackage{type1cm}                    % fontes realmente escaláveis
\usepackage{listings}                   % para formatar código-fonte (ex. em Java)
\usepackage{titletoc}
\usepackage{amsfonts}
\usepackage{amsmath}
\usepackage{clrscode}
%\usepackage[bf,small,compact]{titlesec} % cabeçalhos dos títulos: menores e compactos
\usepackage[fixlanguage]{babelbib}
\usepackage[font=small,format=plain,labelfont=bf,up,textfont=it,up]{caption}
\usepackage[usenames,svgnames,dvipsnames]{xcolor}
\usepackage[a4paper,top=2.54cm,bottom=2.0cm,left=2.0cm,right=2.54cm]{geometry} % margens
%\usepackage[pdftex,plainpages=false,pdfpagelabels,pagebackref,colorlinks=true,citecolor=black,linkcolor=black,urlcolor=black,filecolor=black,bookmarksopen=true]{hyperref} % links em preto
\usepackage[pdftex,plainpages=false,pdfpagelabels,pagebackref,colorlinks=true,citecolor=DarkGreen,linkcolor=NavyBlue,urlcolor=DarkRed,filecolor=green,bookmarksopen=true]{hyperref} % links coloridos
\usepackage[all]{hypcap}                % soluciona o problema com o hyperref e capitulos
\usepackage[square,sort,nonamebreak,comma]{natbib}  % citação bibliográfica alpha (alpha-ime.bst)
\fontsize{60}{62}\usefont{OT1}{cmr}{m}{n}{\selectfont}

% ---------------------------------------------------------------------------- %
% Cabeçalhos similares ao TAOCP de Donald E. Knuth
\usepackage{fancyhdr}
\pagestyle{fancy}
\fancyhf{}
\renewcommand{\chaptermark}[1]{\markboth{\MakeUppercase{#1}}{}}
\renewcommand{\sectionmark}[1]{\markright{\MakeUppercase{#1}}{}}
\renewcommand{\headrulewidth}{0pt}

% ---------------------------------------------------------------------------- %
\graphicspath{{./figuras/}}             % caminho das figuras (recomendável)
\frenchspacing                          % arruma o espaço: id est (i.e.) e exempli gratia (e.g.) 
\urlstyle{same}                         % URL com o mesmo estilo do texto e não mono-spaced
\makeindex                              % para o índice remissivo
\raggedbottom                           % para não permitir espaços extra no texto
\fontsize{60}{62}\usefont{OT1}{cmr}{m}{n}{\selectfont}
\cleardoublepage
\normalsize

% ---------------------------------------------------------------------------- %
% Opções de listing usados para o código fonte
% Ref: http://en.wikibooks.org/wiki/LaTeX/Packages/Listings
\lstset{ %
language=C++,                   % choose the language of the code
basicstyle=\footnotesize,       % the size of the fonts that are used for the code
numbers=left,                   % where to put the line-numbers
numberstyle=\footnotesize,      % the size of the fonts that are used for the line-numbers
stepnumber=1,                   % the step between two line-numbers. If it's 1 each line will be numbered
numbersep=5pt,                  % how far the line-numbers are from the code
showspaces=false,               % show spaces adding particular underscores
showstringspaces=false,         % underline spaces within strings
showtabs=false,                 % show tabs within strings adding particular underscores
frame=single,                 % adds a frame around the code
framerule=0.6pt,
tabsize=2,                      % sets default tabsize to 2 spaces
captionpos=b,                   % sets the caption-position to bottom
breaklines=true,                % sets automatic line breaking
breakatwhitespace=false,        % sets if automatic breaks should only happen at whitespace
escapeinside={\%*}{*)},         % if you want to add a comment within your code
backgroundcolor=\color[rgb]{1.0,1.0,1.0}, % choose the background color.
rulecolor=\color[rgb]{0.8,0.8,0.8},
extendedchars=true,
xleftmargin=10pt,
xrightmargin=10pt,
framexleftmargin=10pt,
framexrightmargin=10pt
}

% ---------------------------------------------------------------------------- %
% Corpo do texto
\begin{document}
\frontmatter 
% cabeçalho para as páginas das seções anteriores ao capítulo 1 (frontmatter)
\fancyhead[RO]{{\footnotesize\rightmark}\hspace{2em}\thepage}
\setcounter{tocdepth}{2}
\fancyhead[LE]{\thepage\hspace{2em}\footnotesize{\leftmark}}
\fancyhead[RE,LO]{}
\fancyhead[RO]{{\footnotesize\rightmark}\hspace{2em}\thepage}

\onehalfspacing  % espaçamento

% ---------------------------------------------------------------------------- %
% CAPA
% Nota: O título para as dissertações/teses do IME-USP devem caber em um 
% orifício de 10,7cm de largura x 6,0cm de altura que há na capa fornecida pela SPG.
\thispagestyle{empty}
\begin{center}
    \vspace*{2.3cm}
    \textbf{
      \Large{
        Projeto Ouroboros \\
      }
      Sistema de integração de C++ com linguagens de \emph{script} \\
    }
    
    \vspace*{1.2cm}
    \Large{Fernando Omar Aluani} \\
    \Large{Wilson Kazuo Mizutani}
    
    \vskip 2cm
    \textsc{Trabalho de conclusão de curso} 
    
    \vskip 10cm
    Orientador: Prof. Dr. Marcos Dimas Gubitoso

    \vskip 3cm
    
    \normalsize{São Paulo, Dezembro de 2013}
\end{center}

\pagenumbering{roman}     % começamos a numerar 

% --------------------------------- %
% Agradecimentos
\chapter*{Agradecimentos}

TODO

% ---------------------------------------------------------------------------- %
% Resumo
\chapter*{Resumo}

TODO

% ------------------------------------------------------------------ %
% facilidades                                                        %
% ------------------------------------------------------------------ %

\def\cyclic#1{\langle #1 \rangle}
\def\script{\emph{script} }
\def\cxx{\verb$C++$ }

% comandos novos %
\newcounter{defcnt}
\newcommand\definicao[2]{
    \stepcounter{defcnt}
    \vspace{0.5cm}
    \vbox{
    \textbf{Definição \thedefcnt \hspace{1cm} #1} \\ #2
    \begin{center}
        $ \square $
    \end{center}
}
}
\newcommand\notacao{
    \textbf{Notação} \hspace{0.2cm}
}

% ---------------------------------------------------------------------------- %
% Sumário
\tableofcontents    % imprime o sumário

% ---------------------------------------------------------------------------- %
% Capítulos do trabalho
\mainmatter

% cabeçalho para as páginas de todos os capítulos
\fancyhead[RE,LO]{\thesection}

\singlespacing              % espaçamento simples

\input cap-objetiva       % associado ao arquivo: 'cap-objetiva.tex'
\input cap-subjetiva      % associado ao arquivo: 'cap-subjetiva.tex'

% cabeçalho para os apêndices
\renewcommand{\chaptermark}[1]{\markboth{\MakeUppercase{\appendixname\ \thechapter}} {\MakeUppercase{#1}} }
\fancyhead[RE,LO]{}
\appendix

%\include{ape-conjuntos}      % associado ao arquivo: 'ape-conjuntos.tex'

% ---------------------------------------------------------------------------- %
% Bibliografia
\renewcommand\bibname{Referências}
\backmatter \singlespacing   % espaçamento simples
\bibliographystyle{alpha-ime}% citação bibliográfica alpha
\bibliography{bibliografia}  % associado ao arquivo: 'bibliografia.bib'

\end{document}
