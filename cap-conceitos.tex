
\chapter{Conceitos Necessários}
\label{cap:conceitos}

  Nosso trabalho lida com os mecanismos que jazem por trás de uma linguagem de
  programação, que a fazem funcionar. Ele inspeciona as informações internas e
  estende o alcance desse mecanismo, de forma que diferentes linguagens consigam
  interagir. Por isso vamos esclarecer a ideia básica do que é uma linguagem de
  programação - na seção \ref{cap:conceitos:linguagens} - e de como programas
  escritos com elas são executados - na seção \ref{cap:conceitos:compiladores}.
  E então, como o sistema que desenvolvemos cai em uma situação semelhante,
  também apresentaremos um par de conceitos computacionais usados no processo de
  analisar o código escrito em uma determinada linguagem - na seção
  \ref{cap:conceitos:gramaticas}.

  \section{Linguagens de Programação}
  \label{cap:conceitos:linguagens}

  Conforme as pessoas percebem os padrões e os protocolos de um procedimento
  qualquer (como escovar os dentes ou dirigir um carro), elas tornam-se capazes
  de descrevê-lo de forma que outros possam reproduzí-lo. E acontece que alguns
  procedimentos (como calcular a média de um conjunto de números ou ordenar uma
  sequência de valores) podem ser descritos de maneira tão exata que podemos
  fazer computadores digitais executá-los para nós.
  
  Obviamente, esses últimos compõem o tipo de processo que é o objeto de
  interesse aqui. São processos chamados de \emph{computações} - justificando o
  termo \emph{computador}. Uma maneira um pouco mais formal de definí-los pode
  ser derivada de \cite[Introduction]{pl:00}:

  \definicao{
    Dada uma coleção de valores e operações, chamada de \textbf{modelo
    computacional}, a aplicação de uma sequência dessas operações sobre um
    desses valores para obter outro valor é o que constitui uma
    \textbf{computação}.
  }

  %Por exemplo, o processador de um computador é responsável por executar
  %operações sobre dados binários armazenados nos registradores ou na
  %memória. Assim, quando esses dados são usados para produzir novos valores,
  %podemos dizer que ocorreu uma \emph{computação}. Nesse caso, o \emph{modelo
  %computacional} seria determinado pelos circuitos lógicos do processador e
  %pelos valores representáveis nas diferentes unidades de armazenamento da
  %máquina.

  % TODO: exemplo com ábaco?

  Da mesma maneira que existem diversas maneiras de se ensinar uma pessoa a
  dirigir um carro, há diferentes métodos de representar uma computação de
  maneira que um computador seja capaz de executá-la. Essa representação, por
  sua vez, precisa de uma \emph{linguagem} através da qual humanos podem
  expressá-la e os computadores podem processá-la. E com isso chegamos ao
  assunto dessa seção:

  \definicao{
    A descrição de uma computação é um \textbf{programa}, e a notação com
    a qual ela é expressa é uma \textbf{linguagem de programação}.
  }

  %Usando novamente o exemplo do computador, as operações que o processador
  %executa seguem uma série de instruções escritas em código de máquina. Essa
  %série de instruções compões um \emph{programa}, e o representação usada para
  %escrevê-las é uma \emph{linguagem de programação} conhecida como
  %\emph{código de máquina}. Tipicamente, ela é baseada em valores puramente
  %numéricos, tornando incrivelmente difícil sua compreensão por humanos.

  % TODO: exemplo com receitas?

  %Como veremos na seção \ref{cap:conceitos:compiladores} a seguir, essas
  %definições também podem ser usadas em outros contextos. Existem outras
  %manifestações de modelos computacionais fora o \emph{hardware} de um
  %computador. E a grande maioria programadores não escreve código de máquina.

  Juntando todos esses elementos, entendemos que se usa uma linguagem de
  programação para expressar um programa, isso é, uma descrição de como computar
  algo dentro de um modelo computacional. Bom, sabemos que existem \emph{muitas}
  linguagens de programação. Cada uma delas tem uma maneira particular de
  satisfazer essa caracterização. Por exemplo, a nível de \emph{hardware}, temos
  a linguagem de código de máquina, cujo modelo computacional é composto pelas
  operações que o processador é capaz de executar somadas aos possíveis valores
  binários que podem ser armazenados na memória disponível.

  Nosso interesse maior está, é claro, nas chamadas linguagens de \script{}.
  Tentaremos, na seção \ref{cap:conceitos:compiladores} a seguir, mostrar os
  aspectos dessas linguagens que exploramos na elaboração do nosso projeto.

  \section{Compiladores e máquinas virtuais}
  \label{cap:conceitos:compiladores}

  O título dessa monografia se refere a ``linguagens de \script{}''. Apesar de
  haver um senso comum entre quais linguagens de programação são, de fato,
  linguagens de \script{}, é difícil expressar claramente o que isso significa.
  Por exemplo, é bastante aceito que \lang{Lua} seja uma linguagem de \script{},
  enquanto que \lang{Java}, não. % TODO: citation needed
  No entanto, tanto uma quanto a outra compilam seus arquivos de código fonte em
  \emph{bytecode}, que por sua vez é processado por suas máquinas virtuais
  implementadas em \C{} ou \CXX{}.

  \section{Gramáticas e analisadores léxico-sintáticos}
  \label{cap:conceitos:gramaticas}

