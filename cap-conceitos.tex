
\chapter{Conceitos e tecnologias estudadas}
\label{cap:conceitos}
  Sempre que encontramos um problema, buscamos uma solução. Muitos dos produtos
  que usamos diariamente são a manifestação dessas soluções que alguém projetou
  para lidar com os empecilhos do dia-a-dia, desde uma mobília para conseguirmos
  sentar e comer até aparelhos elaborados para detectar doenças em nossos
  corpos. Entender o problema que se quer resolver, e saber desprender a solução
  dessa análise é um procedimento fundamental ao projetar esses produtos -- o
  mesmo vale quando desenvolvemos um \textit{software}. Para tanto, existem
  várias maneiras de se modelar os problemas, e uma das mais simples é na forma
  de uma pergunta.

  \emph{Como integrar \CXX{} com linguagens de \script{}?} Um dos primeiros
  conceitos com o qual aspirantes a programadores têm que se habituar é que
  programas são o resultado de um código misterioso escrito usando uma dessas
  chamadas linguagens de programação. A partir desse momento, costuma-se passar
  por aquele período onde se acredita que a quantidade de linguagens dominadas
  define uma métrica das habilidades do programador ou da programadora, e
  ficamos maravilhados pela miríade de linguagens diferentes. Até que entendemos
  que por trás de todas elas a ideia é relativamente a mesma, naquele momento em
  que percebemos que ``ah, então é assim que se faz um laço nessa linguagem''.
  
  Mas anedotas à parte, permanece o fato de que cada linguagem de programação
  funciona através do seu próprio mecanismo de implementação. Como seria
  possível qualquer tipo de integração entre elas, então? Esse capítulo tem
  como objetivo esclarecer essa questão, e com isso destacar o problema que
  o nosso sistema foi projetado para resolver, além do método necessário para
  tanto. Começamos tentando entender melhor o que são esses elementos que
  estamos tentando integrar.

  \section{Linguagens de Programação}
  \label{cap:conceitos:linguagens}
    Uma das coisas mais importantes que todo programador aprende cedo ou tarde é
    que computadores \emph{não são máquinas inteligentes}. Eles fazem apenas e
    exatamente aquilo que mandamos eles fazerem\footnote{A menos de problemas de
    \textit{hardware}, é claro.}. Por outro lado, é isso que torna eles mais
    confiáveis que humanos no que diz respeito a muitas atividades. E o que
    explica essa característica intrínsica dos computadores é que eles operam
    através de instruções \cite{cpp:00}.

    Quando recebemos uma instrução, tipicamente nos está sendo pedido para
    realizar uma ação, possivelmente envolvendo objetos ou outras pessoas. Com o
    computador, as ações que lhe são instruídas envolvem dados
    \cite{org&arch:00}. As transformações e manipulações que ele pode executar
    sobre dados, assim como o que os dados podem representar compõem o que
    chamamos de \textbf{modelo computacional}, pois indicam as possíveis
    \textbf{computações} que podem ser realizadas dentro de tal modelo. Por sua
    vez, dizemos que as intruções que especificam a computação que desejamos que
    o computador faça formam um \textbf{programa}. Finalmente, isso leva ao
    nosso objeto de interesse: para expressar esse programa ao computador usamos
    uma \textbf{linguagem de programação}. De maneira mais formal, podemos
    definir os termos acima conforme \cite{pl:00}:

    \vspace{2em}

    \definicao{
      \vspace{-1.5em}
      \begin{enumerate}
        \item Um \textbf{modelo computacional} é uma coleção de valores e
              operações.
        \item Uma \textbf{computação} é a aplicação de uma sequência de
              operações sobre um valor para obter outro valor.
        \item Um \textbf{programa} é a especificação de uma computação.
        \item Uma \textbf{linguagem de programação} é uma notação para escrever
              programas.
      \end{enumerate}
    }
    
    Essencialmente, linguagens de programação são a maneira que se encontrou
    para deixar claro quais instruções o computador deve seguir, visto que ele
    as seguirá à risca e portanto queremos ter certeza do que lhe estamos
    pedindo. Agora, se queremos integrar linguagens, precisamos entender como
    são seus modelos computacionais e como interagir com eles na prática. No
    caso, estamos mais especificamente interessados em \CXX{} e em linguagens de
    \script{}. Então vamos estudar, nas próximas sessões, as características que
    elas possuem, assim como as possibilidades que elas nos oferecem.

  \section{Da linguagem para a máquina}
  \label{cap:conceitos:maquina}

    O processo que leva do programa escrito à sua execução pelo computador pode
    ocorrer de diversas maneiras. A mais direta é quando a linguagem usada é
    exatamente aquela que o computador foi projetado para processar. Em geral,
    uma linguagem dessas usa apenas valores numéricos para representar
    instruções explícitas de \emph{hardware} para manipulação de dados brutos,
    e são conhecidas como \textbf{linguagem de máquina} \cite{org&arch:00}.
    Para elas, o modelo computacional corresponde às operações que o processador
    é capaz de executar e os dados que podem ser armazenados nos diversos tipos
    de memória da máquina.

    \definicao{
      \textbf{Linguagem de máquina} é toda linguagem de programação que é
      processada diretamente por uma máquina como um computador.
    }
    
    Como é de se imaginar, desenvolver com essas linguagens não é exatamente uma
    tarefa simples. Cada arquitetura de computador possui uma versão própria
    delas, tornando trabalhoso de fazer um mesmo programa compatível com todas
    elas. Além disso, é preciso muitas instruções para realizar computações
    simples, pois as operações que um processador é capaz de fazer são
    normalmente bastante primitivas - como movimentar dados na memória, ou
    comparar o valor dos dados em lugares diferentes dela. Para lidar com esses
    problemas existem dois métodos que essencialmente buscam abstrair as
    computações desejadas em instruções mais abstratas, isso é, em uma linguagem
    de programação mais acessível - ou de alto nível, como se costuma dizer.

    \subsection{Compilação}
    \label{cap:conceitos:maquina:compilacao}

    O primeiro e mais tradicional deles é traduzir programas escritos nessa
    nova linguagem para a linguagem de máquina do computador alvo. E,
    obviamente, não queremos fazer essa tradução nós mesmos, mas sim desenvolver
    um programa que a faça por nós. Programas assim são ditos
    \textbf{compiladores}, pois eles \emph{compilam} a linguagem fonte para o
    linguagem objeto. Nesse caso, também dizemos que a linguagem fonte é uma
    \textbf{linguagem compilada}.
    
    \definicao{
      \vspace{-1.5em}
      \begin{enumerate}
        \item \textbf{Compiladores} são programas que traduzem uma linguagem
              fonte de alto nível em uma linguagem objeto mais próxima de uma
              linguagem de máquina.
        \item Uma \textbf{linguagem de programação compilada} é a linguagem
              fonte de um compilador.
      \end{enumerate}
    }
    
    Em particular, \CXX{} é uma linguagem compilada \cite{cpp:00}. Em suas
    primeiras versões programas escritos nela eram compilados para a linguagem
    \C{}, que por sua vez compilam para linguagens bastante próximas de uma
    linguagem de máquina, conhecidas como linguagens de montagem; atualmente,
    \CXX{} é compilado diretamente para essas linguagens também \cite{cpp:01}.
    Isso significa que uma das linguagens que estamos tentando integrar no fundo
    é basicamente uma linguagem de máquina disfarçada de maneira bastante
    elegante.
    
    Para que mais de um programa escritos em uma linguagem assim
    possam unir suas funcionalidades, é preciso duas coisas. Primeiramente, pelo
    menos um deles tem que conhecer os símbolos\footnotemark{} relevantes do
    outro; ou seja, o código que declara esses símbolos deve estar incluso no
    programa que os usará. A segunda parte envolve, uma vez que os programas
    estejam compilados para o código objeto, ligar esses símbolos importados às
    suas respectivas definições no programa de origem deles. Isso é feito por um
    programa à parte, chamado de \emph{ligador}, ou pelo próprio compilador. A
    maneira que tanto \C{} quanto \CXX{} encontraram para lidar com esse
    procedimento foi fornecer uma divisão para seus arquivos de código fonte:
    uma parte seriam os \emph{arquivos de cabeçalho} - onde os símbolos são
    declarados - e a outra seriam os \emph{arquivos de implementação} - onde a
    definição do que os símbolos significam é feita.

    \footnotetext{
      Esses símbolos são essenciamente nomes que estão associados a partes
      específicas de um programa, como suas variáveis e rotinas.
    }

    Isso permite que as partes de um programa com rotinas interrelacionadas
    possam ser compiladas em programas menores para ser ligados juntos depois,
    e assim apenas parte do programa precisa ser recompilada se só ela for
    modificada. Também fica possível fazer um programa auxiliar que possa ser
    ligado com qualquer outro que precise de suas funcionalidades, que é o que
    chamamos de uma \emph{biblioteca}. A desvantagem de tudo isso são as
    restrições impostas ao processo de desenvolvimento: tanto o compilador
    quanto o ligador (quando necessário) precisam estar à disposição e
    devidamente configurados, e as bibliotecas necessárias devem estar
    acessíveis a eles.

    \subsection{Interpretação}
    \label{cap:conceitos:maquina:interpretacao}

    O segundo método de usar uma linguagem de alto nível para abstrair
    linguagens próximas a linguagens de máquina seria providenciar um programa
    intermediário capaz de simular a execução de instruções. Chamamos esses
    programas de \textbf{interpretadores} \cite{pl:00}. Também dizemos que sua
    linguagem fonte é uma \textbf{linguagem interpretada}. Essencialmente, ao
    invés de tentar levar a linguagem em si para mais perto da máquina, eles
    trazem a máquina para mais perto da linguagem. Isso é, produzem um modelo
    computacional virtual ao invés de serem computadores reais, porém continuam
    igualmente expressivos. Para tanto, eles simulam uma máquina condizente
    dentro de si, constituindo o que é conhecido como uma \textbf{máquina virtual}.

    \definicao{
      \vspace{-1.5em}
      \begin{enumerate}
        \item Uma \textbf{máquina virtual} é uma simulação de um modelo
              computacional dentro de um programa.
        \item \textbf{Interpretadores} são programas que processam uma linguagem
              fonte de alto nível usando uma máquina virtual.
        \item Uma \textbf{linguagem de programação interpretada} é a linguagem
              fonte de um interpretador.
      \end{enumerate}
    }

    Uma das grandes e principais vantagens de linguagens interpretadas é que
    suas abstrações podem produzir muito mais escrevendo muito menos. Outra
    vantagem é que como elas são processadas por um programa durante a execução
    deste, então o código fonte não precisa estar completo imediatamante (como
    ocorre com a compilação). O interpretador pode ir recebendo as intruções uma
    de cada vez, conforme o usuário as digita nas famosas linhas de comando, por
    exemplo. Juntando essas duas  características, é possível um usuário
    utilizar um terminal para acessar seu sistema operacional através de uma
    linguagem interpretada, mas também manter uma coleção de arquivos escritos
    na mesma linguagem implementando algumas rotinas comuns e repetitivas para
    que ele ou ela possa executá-los quando for mais conveniente. São esses
    arquivos auxiliares que chamamos de \textbf{\script{s}} (do inglês,
    ``roteiros'').

    Costumamos dizer que linguagens interpretadas que podem ser usadas
    para escrever \script{s} são \textbf{linguagens de \script{}}. Infelizmente
    não é tão fácil definir formalmente elas, como indica \cite{script:00}. O
    que não chega a ser um problema para nossa discussão, pois as propriedades
    que nos interessam estão na definição das linguagens interpretadas.

    E, com isso, chegamos às outras linguagens que queremos integrar:
    \lang{Lua}\footnote{Versão 5.1, vide \cite{lua:00}} e
    \lang{Python}\footnote{Versão 2.7, vide \cite{python:00}}, ambas
    consideradas linguagens de \script{}. Com o objetivo de encontrar um meio de
    usá-las junto com \CXX{}, fizemos uma pesquisa sobre o funcionamento de suas
    respectivas máquinas virtuais. O que descobrimos, para nossa felicidade, é
    que ambas são bibliotecas desenvolvidas em \C{}, e que seus interpretadores
    nada mais são do que programas que usam essas bibliotecas. Eles cuidam de
    receber o código do usuário, seja através da linha de comando ou de
    \script{s}, e passá-los para a máquina virtual, que os processa devidamente.
    
    Então, obtivemos finalmente um meio de possivelmente resolver o problema
    proposto no começo desse capítulo. \CXX{} talvez possa ser integrado com
    \lang{Lua} ou \lang{Python} se um programa escrito nele for compilado e
    ligado junto com as bibliotecas dessas linguagens, obtendo assim acesso às
    suas respectivas máquinas virtuais. Tendo isso em mente, vamos buscar
    esclarecer o funcionamento dessas bibliotecas na sessão a seguir, antes de
    passarmos para a solução que propusemos no projeto em si.

  \section{As bibliotecas de \lang{Lua} e \lang{Python}}
  \label{cap:conceitos:apis}

    Como vimos na seção \ref{cap:conceitos:maquina:compilacao}, uma das coisas
    que um programa precisa para usar uma biblioteca são os cabeçalhos delas,
    pois eles fornecem os símbolos para as diferentes funcionalidades dela.
    Costumamos dizer que o conjunto desses símbolos declarados em cabeçalhos
    compõem uma interface, pois eles especificam como um programa pode interagir
    com aquela parte do código. No caso de uma biblioteca em \C{} ou \CXX{},
    chamamos isso de uma \textbf{Interface para Programação de Aplicações} ou
    simplesmente \textbf{API} (do inglês, \textit{Application Programming
    Interface})\footnotemark{}.

    \footnotetext{
      Geralmente a ideia de API também vem atrelada a algum tipo de documentação
      que explica como usar a interface. Por exemplo, costuma-se disponibilizar
      uma referência do que cada rotina na biblioteca faz. Além disso, cada
      linguagem de programação tem sua própria maneira de expor uma API, o que
      vemos nessa seção é apenas a que \C{} e \CXX{} usam.
    }

    Essa seção é dedicada a um estudo breve das APIs que as máquina virtuais de
    \lang{Lua} e \lang{Python} oferecem. Ainda com o foco em encontrar meios
    para formar a integração que queremos, explicaremos algumas funcionalidades
    chave dessas APIs.

    \subsection{\lang{Lua}}
    \label{cap:conceitos:apis:lua}

      Sempre que uma aplicação deseja usar uma máquina virtual de \lang{Lua},
      ela precisa usar uma estrutura declarada pela API chamada
      \lang{lua\_State}. Ela representa o estado da máquina e é usada por
      praticamente todas as rotinas da API. O comportamento geral da biblioteca
      abstrai uma pilha: as operações disponíveis alteram o estado da máquina
      inserindo, removendo e manipulando valores em uma estrutura de dados
      contendo uma sequência de valores, dos quais os últimos costumam ser os
      relevantes\footnotemark{}.

      \footnotetext{
        Uma implementação clássica de pilha teria operações que acessam apenas o
        último elemento inserido. Na API \lang{Lua}, existem algumas funções que
        conseguem inserir e remover valores em posições arbitrárias da pilha.
      }

      Esse método tem várias vantagens, dentre elas o fato que o tipo dos
      elementos na pilha não é explícito, permitindo que a linguagem tenha
      tipagem dinâmica \cite{pl:00}. Além disso, o processo para executar
      rotinas fica bastante generalizado: os valores na pilha no começo da
      execução delas são os parâmetros, em ordem, e os valores que ela deixar ao
      final são os resultados devolvidos, também em ordem. Do lado de quem evoca
      a rotina, basta colocar ela mesma na pilha seguida dos parâmetros
      desejados e usar a função \lang{lua\_call()} da API. Por exemplo, suponha
      a seguinte função de potência em \lang{Lua}:
      
      \vspace{1em}

    \begin{lstlisting}[language=lua]
    -- Devolve x elevado a y
    function pow (x,y)
      return x^y
    end
    \end{lstlisting}
      
      \vspace{1em}

      Usando a API para executar essa função, fazemos o seguinte:
      
      \vspace{1em}

    \begin{lstlisting}
    // Indices passado nas chamadas abaixo sao a posicao da pilha na
    // qual queremos aplicar a operacao. Numeros positivos indexam a
    // partir da base da pilha, e negativos do topo para baixo.
    lua_State *L;
    double x, y;
    // ... inicializa o estado e as variaveis ...
    // Vamos supor que a funcao pow eh uma variavel global...
    // A chamada abaixo coloca ela no topo da pilha
    lua_getglobal(L, "pow");
    // Verifica se a funcao realmente existe e eh do tipo certo
    if (lua_isnil(L, -1) || !lua_isfunction(L, -1)
      throw runtime_error("Could not find function <pow>.");
    // Empilha os parametros
    lua_pushnumber(L, x);
    lua_pushnumber(L, y);
    // Evoca a funcao. Ela recebe dois parametros e devolve apenas um valor,
    // como indicado pelos argumentos passados
    lua_call(L, 2, 1);
    // Agora o resultado estah no topo da pilha e o resto foi removido
    double result = lua_tonumber(L, -1);
    \end{lstlisting}
      
      \vspace{1em}

      Esse exemplo também serve para dar uma ideia de como funciona a conversão
      de valores entre a máquina virtual \lang{Lua} e a linguagem
      \C{}\footnotemark{}. As funções com prefixo \lang{lua\_push} da API
      são usadas para converter para a máquina virtual colocando o valor no topo
      da pilha, enquanto que as com \lang{lua\_to} convertem da pilha para
      \C{}. Note que elas \emph{não} removem os elementos da pilha, para isso
      tem a \lang{lua\_pop()}.

      \footnotetext{
        Como a API \lang{Lua} está escrita em \C{}, ela não oferece um meio
        direto de conversão para \CXX{}. No nosso sistema, usamos alguns
        artifícios para contornar isso.
      }

      Um \script{} \lang{Lua} pode ser usado de duas formas pela máquina
      virtual. A maneira mais simples é tratá-lo como uma rotina e simplesmente
      executá-lo, processando as instruções escritas nele em sequência. A outra
      é tratá-lo como um ``módulo'', seguindo a nomenclatura da API. Nesse caso,
      as variáveis e funções definidas nele são agrupadas em um mesma estrutura.
      Por exemplo, um módulo seria assim:
      
      \vspace{1em}

    \begin{lstlisting}[language=lua]
    module "funstuff"

    function guesswhat ()
      while true do
        -- nothing
      end
    end

    \end{lstlisting}
      
      \vspace{1em}

      E ele seria usado da seguinte maneira:
      
      \vspace{1em}

    \begin{lstlisting}[language=lua]
    require "funstuff"

    print "Using a module"
    funstuff.guesswhat()
    \end{lstlisting}
      
      \vspace{1em}

      A linguagem \lang{Lua} em si oferece um tipo básico que representa uma
      tabela de símbolos\footnotemark{}, chamado de \lang{table}. Elas são a
      base de todas as estruturas mais complexas da linguagem. A API fornece
      algumas funções para manipular essas tabelas, além de usar elas para
      alguns mecanismos internos como a armazenagem de variáveis globais e o
      controle de módulos. Também há um mecanismo para atrelar tabelas
      especiais aos valores dentro da máquina virtual para modificar o
      comportamento deles na linguagem (o que permite, por exemplo, construir
      tipos como números complexos compatíveis com as operações aritméticas
      usuais). Essas tabelas especiais são chamadas de \emph{metatabelas}.
      Por outro lado, estruturas em \C{} só podem ser inseridas e extraídas da
      pilha da máquina usando \lang{void*}, o que significa que é preciso tomar
      cuidado, pois o compilador acreditará que o programador sabe o que está
      fazendo. A linguagem chama essas estruturas fornecidas pela linguagem
      nativa de \lang{userdata}.

      \footnotetext{
        \emph{Tabelas de símbolos} são estruturas de dados que costumam abstrair
        duas operações básicas em sua interface: busca e inserção de elementos
        através de \emph{chaves}. Às vezes uma operação de remoção também está
        disponível, mas nem sempre ela é necessária. Em \lang{Lua}, por exemplo,
        basta inserir o elemento nulo em uma chave para remover o que valor
        estava associado a ela.
      }

      Um tipo mais versátil é o tipo \lang{function}, pois é possível passar
      funções em \C{} para a pilha usando ele. Tudo que é preciso é que a
      assinatura usada seja \lang{int <função> (lua\_State *)}, isso é, uma
      função que receba o estado da máquina e devolva um inteiro. Em sua
      implementação, ela deve manipular a pilha de maneira a satisfazer o
      protocolo especificado anteriormente: o que estiver na pilha inicialmente
      deve ser tratado como os parâmetros, e o que for deixado ao final serão
      os resultados, exceto que o valor inteiro devolvido indica quantos
      elementos do topo da pilha de fato devem ser devolvidos. Por exemplo, a
      mesma função \lang{pow} mostrada acima seria assim:

      \vspace{1em}

    \begin{lstlisting}
    #include <cmath>

    int native_pow (lua_State* L) {
      // A funcao luaL_error interrompe a funcao e joga um erro interno na
      // maquina virtual. Costumamos chamar ela com return para deixar
      // claro que nada vai acontecer depois dela.
      if (lua_gettop(L) != 2) // verifica tamanho da pilha
        return luaL_error(L, "Expected 2 parameters but got %d.", lua_gettop(L));
      if (!lua_isnumber(L, 1) || !lua_isnumber(L, 2)) // verifica os tipos
        return luaL_error(L, "Wrong parameter types");
      double  x = lua_tonumber(L, 1),
              y = lua_tonumber(L, 2);
      lua_pushnumber(L, pow(x, y));
      // Nao precisa limpar a pilha, bas avisar que soh o ultimo
      // valor eh para ser devolvido
      return 1;
    }
    \end{lstlisting}

      \vspace{1em}

      Um procedimento similar é usado para fazer módulos nativos, pois eles são
      construídos usando funções nativas como essa. A principal diferença é que
      além de usar a assinatura correta, o nome da função que constrói o módulo
      precisa ter o prefixo \lang{luaopen\_} seguido do nome do próprio módulo.
      Para futuras referências, chamaremos essas funções especiais de
      \emph{funções de inicialização}. Segue um exemplo de construção de um
      módulo ``\lang{native}'' que contém apenas a função \lang{native\_pow}
      que acamos de mostrar:

      \vspace{1em}

    \begin{lstlisting}
    int luaopen_nativemodule (lua_State* L) {
      // O argumento recebido eh o proprio nome do modulo, que
      // no caso vamos ignorar
      lua_settop(L, 0);
      // O modulo serah formado por uma tabela com a funcao que queremos
      lua_newtable(L);
      lua_pushcfunction(L, native_pow);
      // A pilha estah [tabela, funcao]. A chamada abaixo coloca a funcao
      // na tabela usando a chave "pow"
      lua_setfield(L, 1, "pow");
      // O que sobra na pilha eh soh o proprio modulo, entao devolvemos ele
      return 1;
    }
    \end{lstlisting}

      \vspace{1em}

      Para \script{s} poderem usar esse módulo, você precisa compilar ele em uma
      biblioteca dinâmica\footnotemark{} com o nome ``\lang{native.so}'' em um
      Linux ou ``\lang{native.dll}'' em um Windows. Depois é só fazer como
      antes:

      \footnotetext{
        Bibliotecas dinâmicas são compiladas de maneira que o programa que as
        usa não precisa ser completamente ligado com elas na sua própria
        compilação. Em compensação, ele termina de ligar com elas quando é
        executado.
      }

      \vspace{1em}

    \begin{lstlisting}[language=lua]
    require "native"

    print ("2 ^ 10 = " .. native.pow(2,10))
    \end{lstlisting}

      \vspace{1em}

      Um último fato interessante é que a máquina virtual tem coleta automática
      de memória inutilizada. Diferentemente de implementações tradicionais,
      \lang{Lua} não usa contagem de referências, mas sim verificação periódica
      de quais valores ainda são acessíveis. Basicamente, isso significa que não
      há problemas com referências cíclicas.

    \subsection{\lang{Python}}
    \label{cap:conceitos:apis:python}
    Ao contrário de \lang{Lua}, \lang{Python} trabalha internamente com um
    \emph{heap} para guardar seus objetos, porém raramente o usuário precisa
    mexer com o \textit{heap} diretamente. De fato, ao longo do desenvolvimento 
    desse projeto nunca precisamos mexer nele.

    A API \C{} do \lang{Python} é bem compreensiva e documentada, com muitas funções e
    macros para ajudar o usuário, nos diversos aspectos da interação com a
    máquina virtual, como conversão de valores, tratamento de objetos e de
    diversos protocolos que os objetos podem usar, como o protocolo de sequência
    (uma lista ou uma tupla em \lang{Python}, por exemplo, usam esse protocolo) -
    facilitando o uso de sequências pelo código \C{}.
    
    Por exemplo, suponha o seguinte script \lang{Python}:
      
    \vspace{1em}

    \begin{lstlisting}[language=python]
    # teste.py
    def pow(x,y):
        return x**y
    \end{lstlisting}
      
    \vspace{1em}
    
    Para executar a função \lang{pow} em \CXX{} fazemos o seguinte:
      
    \vspace{1em}

    \begin{lstlisting}
    #include <Python.h>
    
    double executaPow(double x, double y) {
        //primeiro importamos o modulo
        PyObject* teste = PyImport_ImportModule("teste"); //new ref
        if (teste == nullptr) {
            // deu erro na importacao do modulo, python setou uma 
            // excecao na API que pode ser vista...
            // entao aqui voce deve ver essa excecao do python,
            // e retornar um valor de erro ou soltar uma excecao de C++
            return 0;
        }
        
        //gracas a facilidade da API, com uma unica chamada podemos
        // 1. pegar uma funcao do modulo
        // 2. construir uma tupla com os parametros a serem passados
        // 3. executar a tal funcao, passando os parametros.
        //e tambem possivel realizar essas operacoes separadamente,
        //cada uma e uma outra chamada de funcao da API.
        PyObject* result = PyObject_CallMethod(teste, "pow", "dd", x, y); //new ref
        if (result == nullptr) {
            // deu erro na chamada de funcao, o mesmo do caso teste==nullptr
            // acima se aplica aqui.
            return 0;
        }
        
        //convertermos o resultado para C
        double resultado = PyFloat_AsDouble(result);
        
        //corrigimos as contagens de referencia
        Py_DECREF(teste);
        Py_DECREF(result);
        
        //e retornamos o resultado
        return resultado;
    }
    \end{lstlisting}
      
    \vspace{1em}
    
    Como é possível notar no exemplo acima, outra facilidade da API é ela usa um sistema 
    primitivo de orientação a objetos em \C{}, e quase tudo em \lang{Python} (um módulo, uma 
    classe, uma função, etc) é representado em \C{} por um ponteiro para \lang{PyObject}, 
    que além de informações sobre o tipo do objeto e outros valores, contém a contagem de 
    referência do objeto. Quando a contagem de referências de um objeto se torna zero, a 
    máquina virtual deleta aquele objeto. E ai está um dos piores pontos da API \lang{Python}: 
    enquanto a máquina virtual gerencia automaticamente as contagens de referência
    quando ela processa código em \lang{Python}, o usuário que usar a API \C{} tem que incrementar e 
    decrementar manualmente as contagens dos objetos que ele está usando, o que é trabalhoso e 
    propenso a erros.
    
    Assim como no Lua, um \script{} pode ser usado de duas formas pela máquina virtual:
    como uma sequência de operações a serem executadas ou como um módulo, contendo definições
    e operações que podem ser usadas por outros módulos. No entanto, no \lang{Python} não há diferença
    entre ambos. Ao importar um \script{} \lang{Python}, ele será executado e quaisquer operações
    declaradas nele serão realizadas, assim como quaisquer definições se tornarão acessíveis
    a quem importou o módulo. Por exemplo, o seguinte script:
    \begin{lstlisting}[language=python]
    from teste import pow
    
    print "2 elevado a 3 eh igual a", pow(2,3)
    \end{lstlisting}
    Irá importar o \script{} teste mostrado acima, que por si só não realiza nenhuma operação -
    ele simplesmente define a função \lang{pow}. Após a importação, esse \script{} não
    define nada, mas realiza uma operação: imprime texto no console, contendo a frase
    ``2 elevado a 3 é igual a'' e o valor devolvido pela execução de \lang{pow(2,3)}.
    
    Também é possível definir variáveis, funções e classes para serem usadas pelos \script{s}
    na linguagem nativa da máquina virtual. Usando a função \lang{pow} como exemplo novamente,
    para implementá-la em \C{} faríamos o seguinte:
    
    \begin{lstlisting}
    #include <Python.h>
    #include <cmath>
    
    PyObject* native_pow (PyObject* self, PyObject* args) {
        double x, y;
        if (!PyArg_ParseTuple(args, "dd", &x, &y)) {
            // - Ocorreu algum erro na conversao dos parametros.
            // Retornar NULL indica para a maquina virtual que ocorreu
            // um erro nesta funcao, e nesse caso nao setamos uma excecao pois
            // a propria funcao PyArg_ParseTuple ja o faz quando ocorre um erro.
            return NULL;
        }
        // Objeto python retornado deve ser sempre uma nova referencia
        return Py_BuildValue("d", pow(x, y) );
    }
    \end{lstlisting}
    
    A assinatura de \lang{native\_pow} é a assinatura de funções nativas de \lang{Python} em \C{},
    também chamadas de \lang{Python C Functions} ou \lang{PyCFunction}. As formas mais comuns
    de habilitar que essa função nativa seja chamada pelos \script{s} são definir algum 
    objeto que contenha ou execute essa função ou definir um módulo nativo. De forma análoga 
    ao \lang{Lua}, para criar um módulo nativo também é necessário inicializá-lo pelo \C{},
    o que usualmente é feito com uma função de inicialização, e também deve seguir uma assinatura
    e ter o prefixo \lang{init}, seguido pelo nome do módulo. Então para implementar o módulo
    \lang{teste} que mostramos anteriormente de forma nativa, extenderíamos o código do
    \lang{native\_pow} com o seguinte:
    
    \begin{lstlisting}
    static PyMethodDef TesteMethods[] = {
        {"pow", native_pow, METH_VARARGS, "pow(x,y) -> retorna x elevado a y."},
        {NULL, NULL, 0, NULL}        /* sentinela */
    };

    PyMODINIT_FUNC initteste(void) {
        Py_InitModule("teste", TesteMethods);
    }  
    \end{lstlisting}
    
    Para usar o módulo, caso você esteja incluindo a máquina virtual de \lang{Python}
    em sua aplicação executável, basta compilar esse arquivo junto com seu programa, 
    e chamar a função de inicialização \lang{initteste} após inicializar a máquina 
    virtual. Caso contrário, basta compilar o módulo numa biblioteca 
    dinâmica - mas não vamos entrar em detalhes neste caso pois para o escopo deste 
    projeto, somente usamos módulos nativos em programas que estão incorporando as 
    máquinas virtuais.
