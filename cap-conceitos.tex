
\chapter{Conceitos Necessários}
\label{cap:conceitos}

  O que significa integrar linguagens de programação? O que faz uma dessas
  linguagens ser uma \textbf{linguagem de \script{}} ou não? Como isso interfere
  no desenvolvimento da ferramenta que apresentamos aqui? É com o intuito de
  responder essas questões que fizemos esse capítulo. Nele, apresentamos a
  teoria e os conceitos por trás do nosso projeto.

  Começamos esclarecendo a ideia geral do que é uma linguagem de programação -
  na seção \ref{cap:conceitos:linguagens} - e de como programas escritos com
  elas podem ser executados - na seção \ref{cap:conceitos:compiladores}.
  E então, uma vez que o sistema que desenvolvemos lida com uma situação
  semelhante, também apresentaremos um par de conceitos computacionais usados
  quando o código escrito em uma determinada linguagem é processado - na seção
  \ref{cap:conceitos:gramaticas}.

  \section{Linguagens de Programação}
  \label{cap:conceitos:linguagens}

  Conforme as pessoas percebem os padrões e os protocolos de um procedimento
  qualquer (como escovar os dentes ou dirigir um carro), elas tornam-se capazes
  de descrevê-lo de forma que outros possam reproduzí-lo. E acontece que alguns
  procedimentos (como calcular a média de um conjunto de números ou ordenar uma
  sequência de valores) podem ser descritos de maneira tão exata que podemos
  mandar computadores digitais realizá-los para nós. Dizemos que um processo
  assim é uma \textbf{computação}, pois ele trabalha apenas sobre um
  \textbf{modelo computacional}.
  
  Em certo ponto, ao invés de ensinarmos um indivíduo a fazer algo, simplesmente
  escrevemos receitas e manuais que especificam como o procedimento deve ser
  feito. No caso dos computadores, isso significa fazer um \textbf{programa}
  que descreve as ações que a máquina deve executar. E da mesma maneira que
  escrevemos receitas e manuais usando uma linguagem natural (como Português ou
  Inglês), programas são expressos através de \textbf{linguagens de
  programação}.

  \definicao{
    De maneira mais formal, podemos definir os termos acima conforme
    \cite[Introduction]{pl:00}:
    \begin{enumerate}
      \item Um \textbf{modelo computacional} é uma coleção de valores e operações.
      \item Uma \textbf{computação} é a aplicação de um sequência de operações
            sobre um valor para obter outro valor.
      \item Um \textbf{programa} é a especificação de uma computação.
      \item Uma \textbf{linguagem de programação} é uma notação para escrever
            programas.
    \end{enumerate}
  }

  E com isso gostaríamos de esclarecer que quando dizemos ``integração entre
  linguagens de programação'' estamos fazendo um abuso de expressão que na
  verdade significa ``integração entre programas escritos em diferentes
  linguagens de programação''. E não ``composição de várias linguagens de
  programação em uma única linguagem''. Além disso, isso mostra que na realidade
  o usuário do nosso sistema estará lidando com \emph{múltiplos} programas que,
  por estarem integrados, comportam-se como se fossem um só.

  Como sabemos, existem inúmeras linguagens de programação. Ao longo dos anos,
  várias classificações foram surgindo como tentativa de diferenciar o
  propósito, o funcionamento e as vantagens de cada uma delas. Cada
  classificação leva em conta um aspecto diferente da linguagem. No que diz
  respeito a esse trabalho, estamos interessados em reconhecer como os programas
  escritos nelas são usados para realizar as computações correspondentes.
  Entendendo esse mecanismo, saberemos se e como é possível interligar a
  execução de programas escritos em diferentes linguagens.

  \section{Compiladores e máquinas virtuais}
  \label{cap:conceitos:compiladores}

  %O título dessa monografia se refere a ``linguagens de \script{}''. Apesar de
  %haver um senso comum entre quais linguagens de programação são, de fato,
  %linguagens de \script{}, é difícil expressar claramente o que isso significa.
  %Por exemplo, é bastante aceito que \lang{Lua} seja uma linguagem de \script{},
  %enquanto que \lang{Java}, não. % TODO: citation needed
  %No entanto, tanto uma quanto a outra compilam seus arquivos de código fonte em
  %\emph{bytecode}, que por sua vez é processado por suas máquinas virtuais
  %implementadas em \C{} ou \CXX{}.

  \section{Gramáticas e analisadores léxico-sintáticos}
  \label{cap:conceitos:gramaticas}

