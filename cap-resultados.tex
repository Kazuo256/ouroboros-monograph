\chapter{Resultados}
\label{sec:resultados}

% TODO: Uma introdução do capítulo, dando uma visão geral dos resultados.

%exemplo de citação
\cite{xtr:00}

\section{Biblioteca \textbf{libouroboros}}
A libouroboros, que é basicamente a biblioteca para \CXX{} contendo o OPA,
já tem praticamente todas as funcionalidades desejadas implementadas e
funcionando normalmente. Contém um sistema para gerenciar as máquinas
virtuais desejadas, para carregar \script{s} independente de sua
linguagem, e uma interface simples, genérica e robusta para
que o usuário de \CXX{} mexa com a máquina virtual e seus \script{s}.

É provável que ainda alteremos algumas coisas nela, arrumando alguma
coisa ou implementando novas funcionalidades. Algo que ainda iremos
fazer é adaptar seu código de conversão de valores de tipos de usuários
para condizer com a interface do OPWIG, apagando todas referências ao
SWIG.

A libouroboros também é facilmente compilada em várias plataformas usando
o CMake para gerar arquivos relativos à compilação, como \textit{Makefile}s
no Linux. As implementações das interfaces abstratas do OPA para as linguagens
de \script{} Python e Lua que fizemos também estão praticamente completas,
sendo distríbuidas juntamente com a libouroboros e sendo compiladas em
bibliotecas diferentes: \emph{libouroboros-python} e \emph{libouroboros-lua},
para Python e Lua, respectivamente. Assim é fácil incorporá-las em um
mesmo programa, basta ligar o programa com a libouroboros e com as 
% TODO: libouroboros-linguagem? seriously?
libouroboros-\textit{linguagem} desejadas. Caso o usuário não queira usar
uma das linguagens, é só não ligar com as bibliotecas correspondentes.

\subsection{Instruções de Uso}
Aqui iremos explicar brevemente o uso da libouroboros e do OPA, desde compilar
(usando \textit{Makefile}) até pegar valores de um \script{}, seguindo um 
simples exemplo de um programa que carrega um \script{}, pega o valor de uma
variável dele e do retorno da execução de uma função.

\subsubsection{Compilação} 
Como ainda não disponibilizamos nenhum pacote
com os binários da libouroboros, você terá que pegar o código fonte dela
e compilá-la manualmente. Para compilar, basta rodar o CMake na pasta 
raiz do projeto para gerar seus arquivos de compilação. Então basta 
executar os comandos de make para compilação:
\begin{verbatim}
  $ cmake .
  $ make libouroboros
  $ make libouroboros-python
  $ make libouroboros-lua
\end{verbatim}
Os dois ultimos comandos são para compilar as implementações para cada linguagem,
e você deve usá-los somente se necessário. Note que apesar de a libouroboros não
ter nenhuma dependência externa, as implementações das linguagens dependem dos
pacotes de desenvolvimento de suas respectivas linguagens de \script{}.

É também possivel simplesmente executar
\begin{verbatim}
  $ cmake .
  $ make
\end{verbatim}
Isso irá compilar a libouroboros, libouroboros-python, libouroboros-lua e o OPWIG
(mais sobre ele na próxima seção). Caso queira, as seguintes opções podem ser
passadas para o CMake para alterar seu comportamento:
\begin{itemize}
  \item OUROBOROS\_CREATE\_BINDINGS: se verdade, o CMake irá tentar criar as 
    implementações de linguagens (padrão é sim).
  \item OUROBOROS\_LUA\_BINDINGS: se verdade, a implementação de Lua será habilitada (padrão é sim).
  \item OUROBOROS\_PYTHON\_BINDINGS: se verdade, a implementação de Python será habilitada (padrão é sim).
\end{itemize}

É possível também compilar os testes unitários da libouroboros, que
criamos para testar suas funcionalidades usando o seguinte comando:
\begin{verbatim}
  $ make ouroboros_test
\end{verbatim}
E para executá-los:
\begin{verbatim}
  $ ./test/ouroboros_test
\end{verbatim}
    
\subsubsection{Ligação} 
Após compilar a libouroboros, compile o seu código fonte e ligue-o estaticamente
com a libouroboros, e com as libouroboros-\textit{linguagem} que preferir, podendo ser qualquer uma
das duas ou ambas.
    
\subsubsection{Exemplo de Código} 
Em ordem, esse exemplo inicializa o \SMgr{}, passando em qual pasta
o \SMgr{} deverá procurar por \script{s}, carrega um \script{},
pega o valor de uma variável do \script{}, seta o valor de outra variável, 
executa uma função pegando o valor de retorno, e finaliza o \SMgr{}:
    
\begin{lstlisting}
#include <opa/scriptmanager.h>
#include <opa/virtualobj.h>

using opa::VirtualObj;
using opa::ScriptManager;

int main() {
    //inicializando
    SCRIPT_MANAGER()->Initialize("./scripts/");
    
    //carregando script
    VirtualObj modulo = SCRIPT_MANAGER()->LoadModule("exemplo");
    
    //pegando variavel
    double valor_antigo = modulo["VariavelNumerica"].value<double>();
    
    //setando novo valor de variavel
    modulo["VariavelNumerica"].set_value<double>(42.0);
    
    //executando funcao
    VirtualObj funcao = modulo["Funcao"];
    VirtualObj retorno = funcao(); 
    // argumentos para a funcao sao passados como VirtualObjs, 
    // e podem ser passados quaisquer quantidade de objetos,
    // ou uma lista de VirtualObj (VirtualObj::List).
    
    //finalizando
    SCRIPT_MANAGER()->Finalize();
    delete SCRIPT_MANAGER();
    
    return 0;
}
\end{lstlisting}

Há uma coisa faltando nesse exemplo. As máquinas virtuais precisam ser registradas no \SMgr{}.
Enquanto os módulos gerados pelo OPWIG fazem isso automaticamente, nesse pequeno exemplo não
usamos nenhum, então temos que fazer isso manualmente. Basta fazer essas modificações no código acima:

\begin{lstlisting}
... [includes anteriores] ...
#include <opa/config.h>
#ifdef OUROBOROS_LUA_BINDINGS
#include <languages/lua/luawrapper.h>
#endif
#ifdef OUROBOROS_PYTHON_BINDINGS
#include <languages/python/pythonwrapper.h>
#endif

using ...

int main () {
#ifdef OUROBOROS_LUA_BINDINGS
    if (SCRIPT_MANAGER()->GetWrapper("Lua") == nullptr)
        SCRIPT_MANAGER()->Register(new opa::lua::LuaWrapper());    
#endif
#ifdef OUROBOROS_PYTHON_BINDINGS
    if (SCRIPT_MANAGER()->GetWrapper("Python") == nullptr)
        SCRIPT_MANAGER()->Register(new opa::python::PythonWrapper());
#endif

    //inicializando
    SCRIPT_MANAGER()->Initialize("./scripts/");
    
    ...
}
\end{lstlisting}

Para esse exemplo funcionar, basta ter um \script{} chamado ``\textit{exemplo}''
na pasta \textit{/scripts/}, na raiz da onde o programa será executado. A extensão
do arquivo deverá ser \textit{.lua} ou \textit{.py} para Lua ou Python respectivamente
(exemplo.: \textit{./scripts/exemplo.py}). Tal \script{} deverá ter uma váriavel
com um valor numérico de ponto flutuante chamada ``VariavelNumerica'', e uma função
chamada ``Funcao'' que não recebe nenhum argumento (no exemplo, o valor de retorno da
função não importa pois não é usado).

Você pode ver a robustez do OPA e de sua generalização das linguagens fazendo o seguinte:
escreva o \script{} em uma linguagem, teste, e então troque o \script{} por um na
outra linguagem, mantendo as restrições ditas acima (nome exemplo,  com uma variável, etc),
e teste novamente. O programa irá executar do mesmo modo que antes, mesmo tendo trocado a
linguagem do \script{} usado, pois a lógica é a mesma e o OPA abstrai as interfaces.


\section{Gerador de \emph{wrappers} \textbf{opwig}}

% TODO: Por que "uma ferramenta externa"?
O OPWIG é uma ferramenta externa com o propósito de substituir o SWIG.
Ela ainda não implementa todas funcionalidades que o SWIG tem, e portanto
ainda não consegue substituir o SWIG completamente, porém já é funcional.
As suas capacidades atuais de geração de \textit{wrappers} são:
\begin{itemize}
  \item \lang{namespaces} são traduzidos em módulos e sub-módulos na máquina
    virtual, de acordo com a estrutura dos \lang{namespaces}.
  \item Variáveis globais de tipos primitivos, possivelmente constantes.
  \item Funções globais que recebam um número qualquer de argumentos de tipos
    primitivos e retorne nada ou um valor primitivo.
  \item Classes simples, com:
    \begin{itemize}
      \item Destrutor.
      \item Construtor trivial.
      \item Atributos de tipos primitivos, possivelmente constantes.
      \item Métodos que recebam um número qualquer de argumentos de tipos
        primitivos e retorne nada ou um valor primitivo.
    \end{itemize}
\end{itemize}

Além disso o código gerado pelo OPWIG também tem um bloco de inicialização
(\textit{bootstrap}) que registra a máquina virtual correspondente no \SMgr{},
caso ela não tenha sido ainda, e registra nela os módulos e sub-módulos que esse
arquivo gerado define.

É importante notar também que o código gerado pelo OPWIG depende do OPA, 
portanto qualquer programa compilado com código gerado pelo OPWIG deve ser
ligado com a libouroboros. Isso é um contraste em relação ao SWIG, o qual gera
código sem nenhuma dependência externa. Esse comportamento foi uma decisão de
projeto que fizemos para simplificar o código gerado pelo OPWIG e sua integração
com a OPA. Supomos que se um usuário está usando o OPWIG ele provavelmente
também usará o OPA, já que nosso sistema provê essas duas vias de comunicação
entre linguagem nativa e linguagens de \script{}, enquanto o SWIG só possibilita
uma delas (a geração de \textit{wrappers}).

\subsection{Instruções de Uso}
Aqui iremos explicar como usar o OPWIG, desde compilação até execução da ferramenta
e integração do código gerado com seu programa.

\subsubsection{Compilação}
Como ainda não disponibilizamos um pacote com a ferramenta pronta, será necessário
compilar o OPWIG. Para tal, basta usar o CMake tal como já explicamos anteriormente:
\begin{verbatim}
  $ cmake .
  $ make opwig
\end{verbatim}
E lembrando que o comando \lang{make} (sem parâmetros) compila o OPWIG juntamente com
as libouroboros.

Note que ao contrário da libouroboros e das libouroboros-\textit{linguagem}, o OPWIG
ainda não foi separado das implementações para Lua e Python de sua interface 
(a especificação de uma linguagem de \script{}). Portanto ele será compilado com 
essas implementações que, assim como as libouroboros-\textit{linguagem} respectivas,
dependem dos pacote de desenvolvimento da sua linguagem de \script{}.

É possível também compilar o extenso conjunto de testes unitários do OPWIG, que
criamos para testar suas funcionalidades usando o seguinte comando:
\begin{verbatim}
  $ make opwig_test
\end{verbatim}
E para executá-los:
\begin{verbatim}
  $ ./test/opwig_test
\end{verbatim}

\subsubsection{Execução}
Após compilado, executar o programa é simples:
\begin{verbatim}
  $ opwig --module-name=NOME ARQUIVO-1 ARQUIVO-2 ... ARQUIVO-N
\end{verbatim}
Onde:
\begin{itemize}
  \item \textbf{NOME}: é o nome do módulo que será gerado.
  \item \textbf{ARQUIVO-i}: são os arquivos de cabeçalho em \CXX{} que contêm a interface
    que você deseja que seja exportada no módulo.
\end{itemize}

Como explicado anteriormente, o OPWIG ainda não está separado das especificações
de linguagens de \script{}, e portanto ao ser executado ele obrigatoriamente
irá gerar um arquivo de código com o módulo para Lua e outro para Python. Ele gera
os arquivos na mesma pasta de onde ele está sendo executado, seguindo o seguinte
padrão de nomeação:
\begin{verbatim}
  <nome da linguagem de script>_<nome do módulo>_wrap.cxx
\end{verbatim}

\subsubsection{Usando os módulos gerados}
Para usar os módulos gerados, basta compilar eles junto com seu programa, ligado com
a libouroboros e com as libouroboros-\textit{linguagem} desejadas.

Qualquer \script{} ou código de uma dada linguagem executado pelo seu programa (como 
carregar um \script{} com o OPA) será capaz de importar o módulo gerado usando o
sistema padrão da linguagem para importação de módulos. E o módulo em si deverá ser
usado como qualquer outro módulo dessa linguagem.
