%% ------------------------------------------------------------------------- %%
\chapter*{Parte Subjetiva}
\label{sec:parte_subjetiva}

\newcommand\materia[3]{\noindent \textbf{#1} - \texttt{#2}\\\indent #3\vspace{0.5cm}\\}

\input cap-subjetiva-omar

\input cap-subjetiva-wil

\chapter{Próximos passos}
\label{sec:proximos_passos}

Para o projeto como um todo:
\begin{itemize}
  \item \emph{\textbf{Scriptception}}: possibilitar uma linguagem de \script{} usar diretamente
    outra linguagem de \script{} facilmente.
  \item \textbf{Mais linguagens}: Adicionar suporte padrão a outras linguagens além de Python e Lua,
    como Ruby, Perl e Octave.
  \item \textbf{Multithread}: Suportar uso multithreaded do Ouroboros.
  \item \textbf{Tutoriais}: Alguns tutoriais sobre como usar as diferentes partes do
    sistema também seria bem útil para ajudar os usuários.
\end{itemize}

Para completar o OPWIG:
\begin{itemize}
  \item \textbf{Pré-processador}: O pré-processador de \C{}/\CXX{} é uma parte 
    importante da linguagem, podendo até definir trechos de código condicionalmente.
    Colocar um pré-processador no \textit{parser} possibilitaria reconhecer
    códigos \CXX{} bem mais complexos. Nós vamos analisar a dificuldade de implementar
    essa funcionalidade, e se acharmos que não gastará muito tempo vamos colocá-la -
    para tal provavelmente vamos procurar alguma ferramenta pronta que ajude nisso.
  \item \textbf{Exportar \lang{Enum}s}: \lang{Enum} é uma estrutura de dados de \CXX{} 
    que representa um tipo e seus possíveis valores. O \textit{parser} do OPWIG já
    é capaz de reconhecer essas estruturas, porém o gerador de código ainda não
    exporta ela para as linguagens de \script{}.
\end{itemize}
