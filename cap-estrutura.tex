\chapter{Estrutura do Projeto}
\label{sec:estrutura}

%TODO citation for metaprogramming
Esse capítulo tem o objetivo de expor o conceito que projetamos para nosso
sistema. Existe uma certa complexidade em entender como os diversos componentes
dele se relacionam - principalmente devido ao fato deles usarem técnicas como
geração de código automatizada e análise reflexiva de código. A ideia por trás
dessas técnicas também serão explicadas nesse capítulo.

A seção \ref{sec:estrutura:geral} buscará deixar
claro o contexto em que o projeto funciona e como as responsabilidades da
solução são distribuídas. As seções \ref{sec:estrutura:opa} e
\ref{sec:estrutura:opwig} detalham as duas principais partes do projeto, e a
seção \ref{sec:estrutura:integration}, por fim, explica como tudo isso funciona
em conjunto para produzir o resultado desejado.

\section{Visão Geral}
\label{sec:estrutura:geral}

Um usuário do nosso sistema estará tipicamente desenvolvendo uma aplicação em
\cxx{} que de alguma forma deve interagir com \script{s}.

\section{Biblioteca de abstração de APIs de linguagens de \emph{script}.}
\label{sec:estrutura:opa}
TODO

\section{Gerador de \emph{wrappers} e interfaces.}
\label{sec:estrutura:opwig}
TODO

\section{Integrando e entregando tudo para o usuário.}
\label{sec:estrutura:integration}
TODO
