%% ------------------------------------------------------------------------- %%
\chapter{Fernando Omar Aluani}
\label{sec:omar_subjetiva}

\begin{framed}
  O que está escrito nesse capítulo agora (versão preliminar da monografia) 
  representa minha opinião sobre o trabalho e o desenvolvimento do mesmo
  até este momento. Acredito que o que está escrito aqui não irá mudar (muito)
  até o término do trabalho pois é minha opinião sobre fatos que já ocorreram.
  Mas para a versão final da monografia eu irei revisar este texto.
\end{framed}

\section{Desafios e Frustrações}
\label{sec:omar:desafios_frustracoes}

Acho que o maior desafio foi mesmo simplesmente a quantidade absurda de trabalho a
ser feito. Mesmo considerando o que já tinhamos feito anteriormente, tivemos muito
mais trabalho ainda a ser feito para substituir o SWIG e implementar as funcionalidades
que queríamos. Desafios relacionados a problemas encontrados, melhor jeito para
implementar algo ou como definir uma boa interface de alguma parte apareciam de
tempos em tempos mas então nós nos reuníamos um dia e facilmente resolviamos ele juntos.

Frustações... Primeiramente a de que demoramos um pouco no primeiro semestre do 
trabalho de formatura a colocar as coisas rodando, e começar a implementar o 
OPWIG. Inicialmente perdemos muito tempo para desenvolver o \textit{parser}
de forma que ele conseguisse reconhecer coisas comuns de serem encontradas
em código \CXX{}. E também demoramos para desenvolver a estrutura básica de 
metadados, como gerá-los e armazená-los.

Outro ponto meio frustante era quando um integrante da dupla atrasava em suas tarefas,
e ai o outro não tinha muito o que fazer (isso aconteceu de ambos lados). O que
ocorreu diversas vezes por falta de tempo, normalmente decorrida por conta de
outras matérias sendo cursadas concorrentemente.

\section{Relação entre o trabalho de formatura e disciplinas do BCC}
\label{sec:omar:relacao_disciplinas_bcc}

Aqui está uma relação de matérias do BCC que mais afetaram o desenvolvimento
deste trabalho de formatura, em minha opinião: \\

\materia{MAC0122}{Princípios de Desenvolvimento de Algoritmos}{
    Como o próprio nome da disciplina já diz, essa matéria foi muito útil
    para aprender a desenvolver algoritmos, usar estruturas básicas de computação
    e a aprender a mexer em C. Com isso e um conhecimento prévio de \lang{C\#} e \lang{Java},
    foi fácil aprender a mexer com \CXX{}, a linguagem principal usada no desenvolvimento
    desse trabalho.
}
\materia{MAC0211}{Laboratório de Programação I}{
    Aqui começamos a aprender a como organizar e modularizar melhor nosso código, simplificando
    o desenvolvimento. E também vimos e aprendemos a mexer com as ferramentas \textit{bison} e 
    \textit{flex}, que foram indispensáveis para o desenvolvimento do \textit{parser} \CXX{}
    usado pelo OPWIG.
}
\materia{MAC0323}{Estrutura de Dados}{
    O conhecimento de estruturas de dados complexas, como criá-las eficientemente, como usá-las
    e até como generalizar partes dessas estruturas foi muito importante no desenvolvimento
    de diversas partes do trabalho.
}
